\section{Лекция 6}

На прошлой лекции была доказано теорема
\begin{theorem}
    Метрическое пространство является нормальным, то есть удовлетворяет аксиомам $T4 + T1$.
\end{theorem}

Вопрос: что нужно добавить для нормального пространства, чтобы оно стало метризуемым?

\subsection{Функциональная отделимость.}

\begin{definition}[Функциональная отделимость]
    Два подмножества $A, B \subset X$ топологического пространства $X$ называются функционально отделимыми в $X$, если существует такая определенная на всём пространстве вещественная ограниченная непрерывная функция $f: X \to \R$, которая принимает во всех точках множества $A$ одно значение $a$, a во всех точках множества $B$ --- некоторое отличное от $a$ значение $b$.
\end{definition}

\begin{definition}
    Подмножество $A \subset X$ топологического пространства $X$ называется всюду плотным в $X$, если $\overline{A} = X$.
\end{definition}

\begin{example}
    Множество $\Q$ всюду плотно в $\R$.
\end{example}

\begin{statement}[Характеризация хаусдорфова пространства]
    $X$ --- хаусдорфово пространство $\Longleftrightarrow$ $\forall x, y \in X, x \neq y$: существует окрестность $O(x) : y \notin \overline{O}(x)$.
\end{statement}
\begin{proof}
    ($\Leftarrow$): $y \notin \overline{O}(x) \Leftrightarrow y \in X \setminus \overline{O}(x)$ --- открыто, значит, существует окрестность $O(y)$: $O(y) \subset X \setminus \overline{O}(x) \ \Leftrightarrow \ O(y) \cap \overline{O}(x) = \es$. Тогда $O(y) \cap \overline{O}(x) = \es$, а значит, $X$ хаусдорфово. \\
    ($\Rightarrow$): От противного: пусть для любой окрестности $O(x): y \in \overline{O}(x)$. Тогда для любой окрестности $V(y): V(y) \, \cap \, O(x) \neq \es$ --- противоречие с хаусдорфовостью $X$. Значит, существует окрестность $O(x) : y \notin \overline{O}(x)$.
\end{proof}

\begin{theorem}[Лемма Урысона]
    Пусть $X$ - нормальное пространство, $A, B$ --- два замкнутых непересекающихся подмножества $X$. Тогда $A$ и $B$ функционально отделимы, т.е. существует непрерывная функция $F: X \rightarrow [0, 1] \subset \R$, такая, что $F(A) = 0$ и $F(B) = 1$.
% todo: вставить картинку про ступенки
\end{theorem}
\begin{proof}
    Для доказательства этой леммы будет использовать двоично-рациональные числа, т.е. множество
    \[
        S = \cbr{q = \frac{m}{2^n} \mid m \in \Z, n \in \N}
    \]
    Двоично-рациональные числа всюду плотны в $\R$ --- доказательство этого факта остаётся читателю в качестве упражнения.
    % todo: добавить картинку для изображения дробно рациональных чисел на [0, 1]

    Стандартное доказательство: 
        
        Будем строить по индукции семейство открытых множеств $U_{q}$, которые мы заиндексируем двоично-рациональными числами $q \in S \cap [0, 1]$:
        
            1. $U_1 = X \setminus B$.
            
            2. $U_0$ должно удовлетворять требованию: $A \subset U_0 \subset \overline{U}_0 \subset U_1$.

            3. $U_{\frac{1}{2}}$ должно удовлетворять требованию: $\overline{U}_0 \subset U_{\frac{1}{2}} \subset \overline{U}_{\frac{1}{2}} \subset U_1$.

            4. $U_{\frac{1}{4}}$ и $U_{\frac{3}{4}}$ должны удовлетворять требованиям: $\overline{U}_0 \subset U_{\frac{1}{4}} \subset \overline{U}_{\frac{1}{4}} \subset U_{\frac{1}{2}}$, \ \ $\overline{U}_\frac{1}{2} \subset U_{\frac{3}{4}} \subset \overline{U}_{\frac{3}{4}} \subset U_{1}$.

            5. Индуктивный переход: на n-ом шаге берём $q = \frac{2k + 1}{2^n}$. Рассматриваем соседние c $q$ двоично-рациональные числа: $\frac{k}{2^{n - 1}}$ и $\frac{k + 1}{2^{n - 1}}$. Тогда ${U}_{q}$ должно удовлетворять требованию:
            \[
                \overline{U}_{\frac{k}{2^{n - 1}}} \subset U_{\frac{2k + 1}{2^n}} \subset \overline{U}_{\frac{2k + 1}{2^n}} \subset U_{\frac{k + 1}{2^{n - 1}}}.
            \]

            Заметим, что все данные множества существуют в силу нормальности пространства $X$. Поясним это подробнее:
            Характеризация $T_3$ и $T_4$-пространств: \\
            $T_3$): Если $x_0 \in X$, $F \subset X$ --- замкнуто и не содержит точку $x_0$, то существует окрестность $O(x_0): \ \overline{O}(x_0) \cap F = \es$, \\
            $T_4$): Если $F_1, F_2$ --- замкнутые непересекающиеся подмножества $X$, то существует окрестность $O(F_1): \ \overline{O}(F_1) \cap F_2 = \es$. \\
            Доказательства этих утверждений аналогичны доказательству утверждения о характеризации хаусдорфова пространства.

            Тогда: на первом шаге: $U_1 = X \setminus B$. $B$ замкнуто, значит, $U_1$ открыто. Пользуясь приведёнными утверждениями, положим $U_0 = O(A): \overline{U_0} \cap B = \es$. Тогда $\overline{U_0} \subset X \setminus B = U_1$.  \\
            В индукционном переходе: положим $P = \overline{U}_{\frac{k}{2^{n - 1}}}, \ Q = X \setminus U_{\frac{k + 1}{2^{n - 1}}}$. Тогда: существует такая окрестность $O(P)$, что $\overline{O}(P) \cap Q = \es$. Значит, $\overline{O}(P) \subset X \setminus Q = U_{\frac{k+1}{2^{n-1}}}$. Окрестность $O(P) = O(\overline{U}_{\frac{k}{2^{n - 1}}})$ мы и берём в качестве множества $U_{\frac{2k+1}{2^{n}}}$ --- оно удовлетворяет требованиям, указанным в построении. Т.о. все строящиеся множества существуют.
            
            Итак, мы построили систему открытых множеств $\cbr{U_q}$, причём она естественно упорядочена по включению: если $q, r \in S \cap [0, 1]$ и $q < r$, то $\overline{U}_{q} \subset U_{r}$. Теперь определим функцию $F: X \to [0,1]$: \\

        \[
            F(x) = 
                \begin{cases}
                    \inf\cbr{q \mid x \in U_q}, & x \notin B, \\
                    1, & x \in B.
                \end{cases}
        \]

        Очевидно, $F\!\mid_{A} \ \equiv 0$ и $F\!\mid_{B} \ \equiv 1$.
        Проверим непрерывность. По замечению (З.\,\ref{NB:Sufficient_condition_for_continuity_in_terms_of_base}) на стр.~(\pageref{NB:Sufficient_condition_for_continuity_in_terms_of_base}) достаточно проверить, что $F^{-1}(O_\alpha)$ --- открытое множество, где $O_\alpha$ --- элементы предбазы топологии отрезка $[0, 1]$. Т.о. достаточно рассмотреть $O_{\alpha}$ равными $[0, \alpha)$ или $(\alpha, 1]$ для всех $\alpha \in [0,1]$. 

            1) Рассмотрим $x \in F^{-1}([0, a)) \ \Leftrightarrow \ F(x) < a \ \Leftrightarrow \ \inf\cbr{q \mid x \in U_q} < a \ \Leftrightarrow \ \exists \, \widetilde{q}: x \in U_{\widetilde{q}}, \ \widetilde{q} < a$. Тогда $F^{-1}([0, a)) = \bigcup_{\widetilde{q} < a} U_{\widetilde{q}}$ \ --- открыто как объединение открытых. \\
            2) Заметим, что данное выше определение $F$ равносильно следующему: 
            \[
                F(x) = 
                    \begin{cases}
                        \sup\cbr{r \mid x \notin U_r}, & x \notin B, \\
                        1, & x \in B.
                    \end{cases}
            \]
            Тогда: при $x \not\in B$: $F(x) = \sup\cbr{r \mid x \notin U_r} = \sup\cbr{r \mid x \notin \overline{U}_r} = \sup\cbr{r \mid x \in X \setminus \overline{U}_r}$, множество $X \setminus \overline{U}_r$ открыто. Рассмотрим $x \in F^{-1}((b, 1]) \ \Leftrightarrow \ F(x) > b \ \Leftrightarrow \ \sup\cbr{r \mid x \in X \setminus \overline{U}_r} > b \ \Leftrightarrow \ \exists \, \widetilde{r}: x \in X \setminus \overline{U}_{\widetilde{r}}, \ \widetilde{r} > b$. Тогда $F^{-1}((b, 1]) = \bigcup_{\widetilde{r} > b} \br{X \setminus \overline{U}_{\widetilde{r}}}$ \ --- открыто как объединение открытых. \\
            Т.о. построили функцию $F$, удовлетворяющую условиям теоремы.
    % todo: вставить иллюстрацию.
\end{proof}

\subsection{Связь компактности и нормальности.}

\begin{statement}
    Замкнутое подмножество компакта --- компакт. Обратное, вообще говоря, неверно.
\end{statement}
\begin{proof}
    Доказательство данного утверждения было оставлено в качестве упражнения в лекции 4. 
\end{proof}

\begin{exercise}
    Доказать, что $\overline{A \cup B} = \overline{A} \cup \overline{B}$.
\end{exercise}

\begin{lemma}
    В хаусдорфовом топологическом пространстве $X$ компактное подмножество $F$ является замкнутым.
\end{lemma}
\begin{proof}
    Пусть $F$ не замкнуто. Тогда $\exists x_0 \in \overline{F}: x_0 \notin F$. Пусть $y \in F$. Т.к. $X$ хаусдорфово, то существует окрестность: $O(y): x \notin \overline{O}(y)$. Рассмотрим открытое покрытие $\bigcup_{y \in F} O(y)$ множества $F$ такими окрестностями. Т.к. $F$ --- компакт, то существует конечное подпокрытие $\bigcup_{i = 1}^{n} O(y_i) \supset F$. Тогда
    \[
        \overline{\bigcup_{i = 1}^{n} O(y_i)} \supset \overline{F} \ \Longleftrightarrow \ \bigcup_{i = 1}^{n} \overline{O(y_i)} \supset \overline{F}.
    \]
    Но $\forall i \in \cbr{1, \ldots, n}: x_0 \notin \overline{O(y_i)}$, значит, $x_0 \notin \bigcup_{i = 1}^{n} \overline{O(y_i)}$, при этом $x_0 \in \overline{F} \subset \bigcup_{i = 1}^{n} \overline{O(y_i)}$ --- противоречие. Значит, $F$ замкнуто.
\end{proof}