\section{Лекция 6}

На прошлой лекции была доказано теорема
\begin{theorem}
    Метрическое пространство является нормальным, то есть удовлетворяет аксиомам $T4 + T1$.
\end{theorem}

Вопрос: что нужно добавить для нормального пространства, чтобы оно стало метризуемым?

\subsection{Функциональная отделимость}

% \begin{definition}[функциональная отделимость]
    
% \end{definition}

\begin{definition}
    $A \subset X$ - всюду плотно в $X$, если $\overline{A} = X$.
\end{definition}

\begin{theorem}[Лемма Урысона]
    Пусть $X$ - нормальное пространство. $A, B$ - два замкнутых непересекающихся подмножества $X$. Тогда существуеют непрерывная функция $F: X \rightarrow [0, 1] \subset \R$, такая, что $F(A) = \cbr{0}$ и $F(B) = \cbr{1}$
% todo: вставить картинку про ступенки
\end{theorem}
\begin{proof}
    Для доказательства этой леммы будет использовать двоично-рациональные числа, это
    \[
        S = \cbr{q = \frac{m}{2^n}, m \in \Z, n \in \N}
    \]
    % todo: добавить картинку для изображения дробно рациональных чисел на [0, 1]

    Стандартное доказательство: 
        
        Будем строить по индукции семейство открытых множеств $U$, которые мы заиндексируем двоично-рациональными числа из $[0, 1]$.
        
            1. $U_1 = X \subset B$
            
            2. $U_0$ должно быть следующим $A \subset U_0 \subset \overline{U}_0 \textbf{ (используем нормальность) } \subset U_1$

            3. $U_{\frac{1}{2}}$ должно выполняться $\overline{U}_0 \subset U_{\frac{1}{2}} \subset \overline{U}_{\frac{1}{2}} \subset U_1$, cуществование такого множества следует из нормальности, примененной к дополнениям $U_0$ и $U_1$.

            4. $U_{\frac{1}{4}}:$ $\overline{U}_0 \subset U_{\frac{1}{4}} \subset \overline{U}_{\frac{1}{4}} \subset U_{\frac{1}{2}}$ и $U_{\frac{3}{4}}:$ $\overline{U}_\frac{1}{2} \subset U_{\frac{3}{4}} \subset \overline{U}_{\frac{3}{4}} \subset U_{1}$

            5. индуктивный переход. Берем $q = \frac{2k + 1}{2^n}$. Рассмотрим соседние c $q$ столбики - они будут иметь вид $\frac{k}{2^{n - 1}}$ и $\frac{k + 1}{2^{n - 1}}$.
            \[
                \overline{U}_{\frac{k}{2^{n - 1}}} \subset U_q \subset \overline{U}_q \subset U_{\frac{k + 1}{2^{n - 1}}}
            \]

        Пострили системс открытых множеств. Это система множеств $\cbr{U_q}$ обладает свойством упорядоченности, т.е. если $q < r \in S$, то $\overline{U}_q \subset U_r$. Теперь определим функцию $F$.

        % todo: добавить оператор inf

        \[
            F(x) = 
                \begin{cases}
                    \inf\cbr{q : x \in U_q}, & x \notin B \\
                    1, & x \in B
                \end{cases}
        \]

        Очевидно выполнение требований для множеств $A$ и $B$.
        Проверим непрерывность. Достаточно проверить, что $F^{-1}(O_\alpha)$ - открыт, где $O_\alpha$ - элемент базы топологии отрезка. Можно это доказать только для $[0, a), (b, 0]$, т.к. остальные элементы топологии можно получить из этих двух.

            Рассмотрим $x \in F^{-1}([0, a)) \Leftrightarrow F(x) < a  \Leftrightarrow \inf\cbr{q : x \in U_q} < a \Leftrightarrow \exists \widetilde{q} < a$, тогда $F^{-1}([0, a)) = \bigcup_{\widetilde{q} < a} U_{\widetilde{q}}$ - открыто.

            Рассмотрим $F$, заданную другим образом 
            \[
                F(x) = 
                    \begin{cases}
                        \sup\cbr{r : x \notin U_r}, & x \notin B \\
                        1, & x \in B
                    \end{cases}
            \]
            $\sup\cbr{r : x \notin U_r} = \sup\cbr{r : x \notin \overline{U}_r} = \sup\cbr{r : x \in X \setminus \overline{U}_r - \text{ это множество открыто}}$
            Далее аналогично первому случаю.
            
    Иллюстрация:
    % todo:
\end{proof}

\begin{example}[Нормального, но не метризуемого пространства]
    
\end{example}

\subsection{Взаимоотношение компактности и нормальности}

\begin{nota_bene}[характеризация хаусдорфово пространства]
    Пусть $X$ - хаусдорфово $\Leftrightarrow$ для каждых $x \neq y$ существует $O(x) : y \notin \overline{O}(x) $
\end{nota_bene}
\begin{proof}
    ($\Leftarrow$): $y \notin \overline{O}(x) \Leftrightarrow y \in X \setminus \overline{O}(x)$ - открыто, тогда существует окрестность $O(y)$: $O(y) \cap \overline{O}(x) = \emptyset$, тогда $O(y) \cap \overline{O}(x) = \emptyset$.

    % todo:
    ($\Rightarrow$): от противного 
\end{proof}

\begin{statement}
    Замкнутое подмножество компакта - компактно
\end{statement}
\begin{proof}
    Очевидно.
\end{proof}

\begin{lemma}
    В хаусдорфовом топологическом пространстве $X$ компактное подмножество $F$ является замкнутым.
\end{lemma}
\begin{proof}
    Очевидно.
\end{proof}

\begin{exercise}
    $\overline{A \cup B} = \overline{A} \cup \overline{B}$
\end{exercise}
