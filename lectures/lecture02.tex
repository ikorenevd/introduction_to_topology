\section{Лекция 2}

Литература:
\begin{enumerate}
    \item Федорчук В.В., Филиппов В.В. --- Общая топология. Основные конструкции.
    \item Виро О.Я., Иванов О.А., Нецветаев Н.Ю., Харламов В.М. --- Элементарная топология.
\end{enumerate}

\begin{definition}
    Пусть $(X, \mcT)$ --- топологическое пространство. Если топология $\mcT$ на $X$ может быть порождена некоторой метрикой $\rho$ на $X$, то пространство $(X, \mcT)$ называется метризуемым.
\end{definition}

\begin{nota_bene}
    Существует ряд критериев метризуемости топологических пространств: см. Критерий метризуемости Нагаты --- Ю.М.Смирнова, 1950-1951.
\end{nota_bene}

\begin{definition}
    Пусть $(X, \rho)$ --- метрическое пространство. Открытый шар радиуса $\veps > 0$ с центром в точке $x_0$ --- это множество $O_{\veps}(x_0) = \cbr{x \in X : \rho(x, x_0) < \veps}$.
\end{definition}

\begin{theorem}\label{T:Base_of_Metric}
    Пусть $(X, \rho)$ --- метрическое пространство. Тогда шары $O_{\veps}(x)$ образуют базу топологии, порождённой на $X$ метрикой $\rho$.
\end{theorem}
\begin{proof}
    Рассмотрим множество $\mfB$ всех открытых шаров в пространстве $X$: \quad $\mfB = \cbr{O_{\veps}(x) \mid x \in X, \ \veps > 0}$. Проверим для $\mfB$ оба пункта достаточного условия на базу: \\
    1. Очевидно, $\forall x \in X \ \exists \, O_{\veps}(x) : x \in O_{\veps}(x)$ \\
    2. Обозначим: $B_1 = O_{\veps_1}(x_1), \ B_2 = O_{\veps_2}(x_2)$ и покажем, что $\forall x \in B_1 \cap B_2 \quad \exists B_3 = O_{\veps}(x) \in \mfB: B_3 \subset B_1 \cap B_2$. По определению открытых шаров $B_1$ и $B_2$:
    $\rho(x, x_1) < \veps_1, \ \rho(x, x_2) < \veps_2$. Положим $\veps = \min\cbr{\frac{\veps_1}{2} - \rho(x_1, x), \frac{\veps_2}{2} - \rho(x_2, x)}$. Тогда: $\forall y \in O_{\veps}(x)$:
    % todo: вставить картинку
    \[
        \rho(y, x_1) \leq \rho(y, x) + \rho(x, x_1) = \veps + \rho(x, x_1) \leq \frac{\veps_1}{2} - \rho(x, x_1) + \rho(x, x_1) = \frac{\veps_1}{2} < \veps_1.
    \]
    Значит, $y \in O_{\veps_1}(x_1)$. Аналогично: $\rho(y, x_2) < \veps_2 \Rightarrow y \in O_{\veps_2}(x_2)$. Т.к. это верно $\forall y \in O_{\veps}(x)$, то: $O_{\veps}(x) \subset B_1 \cap B_2$, т.е. $B_3 \subset B_1 \cap B_2$.
    Т.о. по достаточному условию на базу топологии: открытые шары в метрическом пространстве образуют базу топологии, порождённой метрикой этого пространства.
\end{proof}

\begin{definition}
    Пусть на множестве $X$ заданы две топологии $\mcT_1$ и $\mcT_2$. Говорят, что $\mcT_2$ сильнее $\mcT_1$ \ ($\mcT_1$ слабее $\mcT_2$) и пишут $\mcT_1 \leq \mcT_2$, если $\mcT_1 \subseteq \mcT_2$, т.е. если любое открытое в $\mcT_1$ множество будет открытым в $\mcT_2$.
\end{definition}
Такой способ сравнения топологий на множестве $X$ относительно прост. Введённое отношение сравнения является отношением частичного порядка и образует на множестве всех топологий на $X$ структуру частично упорядоченного множества (ЧУМа).

\begin{example}
    Рассмотрим антидискретную и дискретную топологии на множестве $X$: 
    \[
        \mcT_1 = \cbr{\es, X} \subset \mcT_2 = 2^{X}.
    \]
    В некотором смысле это два полюса сравнения: антидискретная топология на $X$ является слабейшей, а дискретная --- сильнейшей, т.е. для любой топологии $\mcT$ на $X$: \, $\mcT_1 \leq \mcT \leq \mcT_2$. Тем не менее введённый порядок на $X$ является частичным, и нетривиальные топологии могут быть несравнимы.
\end{example}

\begin{exercise}
    Метризумы ли тривиальные топологии(= антидискретная и дискретная)?
    Ответ:
    \begin{enumerate}
        \item Рассмотрим дискретную метрику: $\rho_{D}(x, y) =
        \begin{cases}
            1, &\text{если $x \neq y$,} \\
            0, &\text{если $x = y$.}
        \end{cases} $
        Дискретная метрика порождает дискретную топологию.
        \item Антидискретная топология неметризуема.
    \end{enumerate}
\end{exercise}

\begin{definition}[Индуцированной топологии подространства]
    Пусть $(X, \mcT)$ - топологическое пространство, $Y \subset X$. Тогда $Y$ образует топологическое пространство с топологией, называемой индуцированной (с пространства $X$) топологией (топологией ограничения) \ $\mcT\!\mid_{Y} \ = \, \cbr{Y \cap U \mid U \in \mcT}$.
\end{definition}
\begin{exercise}
    Проверить, что индуцированная топология действительно является топологией на множестве $Y$, т.е. удовлетворяет аксиомам из определения топологии.
\end{exercise}

\begin{example}
    $X = \R^2$ - метрическое пространство с евклидовой метрикой, $Y \subset X$. Базой топологии, порождённой метрикой на пространстве $X$, являются открытые шары, а базой индуцированной топологии на $Y$ являются всевозможные пересечения открытых шаров в $X$ с $Y$.
    % todo: вставить картинку 
\end{example}

\begin{definition}
    Окрестность точки $x$ в топологическом пространстве --- это любое открытое множество этого пространства (т.е. элемент топологии), содержащее $x$.
\end{definition}

\begin{nota_bene} Из определений топологии и окрестности точки очевидно следует, что:
    \begin{enumerate}
        \item Пересечение конечного числа окрестностей точки является её окрестностью,
        \item Объединение (произвольного числа) окрестностей точки является её окрестностью.
    \end{enumerate}
\end{nota_bene}

\begin{statement}
    Пусть $(X, \mcT)$ --- топологическое пространство. Тогда \ $A \subseteq X$ - открыто $\Leftrightarrow$ для каждой точки $x \in A$ существует её окрестность, лежащая в $A$.
\end{statement}
\begin{proof}
    % todo: вставить картинку
    
    $\br{\Leftarrow}$: По условию: $\forall x \! \in \! A \ \exists \, O(x) \! \in \! \mcT: x \in O(x), \, O(x) \subseteq A$. Рассмотрим $C = \bigcup_{x \in A} O(x)$: \ $C \in \mcT$. 
    Очевидно, что $A \subseteq C$, а т.к. для каждого $x \in A$ верно $O(x) \subseteq A$, то $C \subseteq A$. Получаем, что $A = C$, значит, $A \in \mcT$.

    $\br{\Rightarrow}$: Раз $A$ открыто, то $A$ является окрестностью любой своей точки.
\end{proof}

\begin{definition}
    Пусть $x \in X$. Если $\cbr{x} \in \mcT$, то $x$ называется изолированной точкой пространства $X$.
\end{definition}

\begin{nota_bene}
    В дискретной топологии на любом пространстве все точки являются изолированными.
\end{nota_bene}

\begin{definition}
    Пусть $x \in X$, $A \subset X$. Тогда $x$ называется точкой прикосновения множества $A$, если для любой её окрестности $O(x)$ выполняется $O(x) \cap A \neq \es$. 
\end{definition}

\begin{definition}
    Пусть $x \in X$, $A \subset X$. Тогда $x$ называется внутренней точкой множества $A$, если существует её окрестность $O(x)$: $O(x) \subset A$.
\end{definition}

\begin{definition}[A1]
    Замыкание множества $A$ --- это множество всех точек прикосновения $A$.
    Обозначение: $\overline{A}$.
\end{definition}

\begin{definition}[B1]
    Внутренность множества $A$ --- это множество всех внутренних точек $A$.
    Обозначение: $\Int(A)$.
\end{definition}

\begin{exercise}
    Показать, что: $\Int(A) \subset A \subset \overline{A}$.
\end{exercise}

\begin{definition}[A2]
    Замыкание $\overline{A}$ множества $A$ --- это пересечение всех замкнутых множеств, содержащих $A$. Иными словами,
    $\overline{A}$ --- это наименьшее по включению замкнутое множество, содержащее $A$.

\end{definition}

\begin{definition}[B2]
    Внутренность $\Int(A)$ множества $A$ --- это объединение всех открытых множеств, лежащих в $A$. Иными словами,
    $\Int(A)$ --- это наибольшее по включению открытое множество, лежащее в $A$.
\end{definition}

\begin{theorem} % todo: дописать доказательство
    Определение A1 эквивалентно определению A2; Определение B1 эквивалентно определению B2.
\end{theorem}
\begin{proof}
    Доказательство эквивалентности определений A1 и A2 остаётся в качестве упражнения читателю.
    Докажем эквивалентность определений B1 и B2. \\
    Пусть $\Int_1(A)$ - множество внутренних точек $A$ в смысле определения B1, а $\Int_2(A)$ --- в смысле определения B2. Покажем, что эти множества равны:

    $(\subseteq)$: Если $x \in \Int_1(A)$, то существует его окрестность $O(x) \subset A$. Но $O(x)$ --- открыто, а значит, $O(x) \subset \Int_2(A)$, и $x \in \Int_2(A)$. Получаем, что $\Int_1(A) \subseteq \Int_2(A)$.
    
    $(\supseteq)$: Если $x \in \Int_2(A)$, то $x$ принадлежит какому-то открытому $V \subset A$. Но тогда мы можем взять $V$ в качестве окрестности точки $x$. Получаем, что $x \in \Int_1(A)$, а значит, $\Int_1(A) \supseteq \Int_2(A)$.
    Итак, $\Int_1(A) = \Int_2(A)$, а значит, определения B1 и B2 эквивалентны.
\end{proof}

\begin{definition}
    Пусть $x \in X$, $A \subset X$. Тогда $x$ называется граничной точкой множества $A$, если $x$ является точкой прикосновения $A$, но не является внутренней точкой $A$, т.е. если \ $x \in \overline{A}, \ x \notin \Int(A)$.
\end{definition}

\begin{definition}
    Граница множества $A$ --- это множество всех граничных точек $A$. Обозначение: $\Bd(A)$ или $\partial A$.
    По определению: $\Bd(A) = \overline{A} \setminus \Int(A)$.
\end{definition}

% todo: добавить картинку
\begin{definition}[Понятия непрерывного отображения]
    Пусть $(X, \mcT_{X})$, $(Y, \mcT_{Y})$ --- топологические пространства, $f: X \rightarrow Y$. Отображение $f$ называется непрерывным в точке $x_0 \in X$, если для любой окрестности $O(f(x_0)) \in \mcT_{Y}$ существует такая окрестность $U(x_0) \in \mcT_{X}$, что $f(U(x_0)) \subset O(f(x_0))$. 

    Отображение $f$ называется непрерывным (непрерывным отображением топологических пространств), если оно непрерывно во всех $x \in X$.
\end{definition}

% todo: дописать доказательство
\begin{statement} %bigchange! Переместил сюда доказательство из начала 3 лекции --- оно лаконичней, зачем повторяться
    Следующие условия эквивалентны:
    \begin{enumerate} %needsbigchange! Надо что-то сделать с пунктом 4: либо убрать заметку о лекции, либо привести контрпример
        \item Отображение топологических пространств $f: X \rightarrow Y$ непрерывно.
        \item Прообраз любого открытого множества под действием $f$ является открытым, т.е. $U \in \mcT_{Y} \Rightarrow f^{-1}(U) \in \mcT_{X}$.
        \item Прообраз любого замкнутого множества под действием $f$ является замкнутым.
        \item Для любого $A \subseteq X$: $f(\overline{A}) \subseteq \overline{f(A)}$ (На лекции утверждалось не включение, а равенство, но это неверно).
    \end{enumerate}
\end{statement}
\begin{proof}
    Доказательство эквивалентности условий 1, 3 и 4 остаётся в качестве упражнения читателю. \\
    Докажем $(1) \Leftrightarrow (2)$:

    $(\Rightarrow)$: Пусть $V \subset Y$ открыто. Рассмотрим $\forall x \in f^{-1}(V)$: Т.к. $V \in \mcT_{Y}$ и $f(x) \in V$, то $\exists \, O(f(x)) \subset V$ --- окрестность $f(x)$. Т.к. $f$ непрерывно, то для найденной $O(f(x)) \ \exists \, U(x) \in \mcT_{X}$ --- окрестность $x: f(U(x)) \subset O(f(x)) \subset V$.
    Значит, $U(x) \subset f^{-1}(V)$. Получаем, что любая точка из $f^{-1}(V)$ входит в это множество вместе с некоторой своей окрестностью, а значит, $f^{-1}(V)$ открыто. Итак, прообраз любого открытого множества под действием $f$ открыт.

    $(\Leftarrow)$: Пусть $x \in X$. Рассмотрим $\forall \, O(f(x)) \in \mcT_{Y}$ --- окрестность $f(x)$. Т.к. $O(f(x))$ открыто, то $f^{-1}(O(f(x)))$ открыто в $X$ --- выберем это множество в качестве окрестности $x$. Получаем, что $\forall x \in X \ \forall \, O(f(x)) \in \mcT_{Y} \ \exists U(x) \in \mcT_{X}: f(U(x)) \subset O(f(x))$, т.е. $f$ непрерывно.
\end{proof}