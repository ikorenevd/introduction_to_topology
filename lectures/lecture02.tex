\section{Лекция 2}

Литература:
\begin{enumerate}
    \item В.В. Федорчук - Введение в топологию
    \item 4 автора - Введение в топоплогию
\end{enumerate}

\begin{definition}
    Пусть $(X, \rho)$ - метрическое пространство. Из метрического можно построить топологические пространство $(X, \tau)$.
    Такие пространства называеются метризуемыми.
\end{definition}

\begin{nota_bene}[Критерий метризуемости. Накаты Ю.М.Смирнова. 1950-1951]
\end{nota_bene}

\begin{definition}
    $O_{\varepsilon}(x_0) = \cbr{x\in X : \rho(x, x_0) < \varepsilon}$
\end{definition}

\begin{theorem}
    Шары $O_{\varepsilon}(x)$ образуют базу топологии.
\end{theorem}
\begin{proof}
    1. $\forall x \in X \ \exists \ O_{\varepsilon}(x) : x \in O_{\varepsilon}(x)$

    2. рассмторим $(O_{\varepsilon_1}(x_1) = B_1) \cap (B_2 = O_{\varepsilon_2}(x_2))$ найдем окрестность точки $x$, лежащий в этом пересечении, чтобы вся окрестность тоже лежала в этом пересечении
    $\rho(x, x_1) < \varepsilon_1, \rho(x, x_2) < \varepsilon_2$. Пусть $\varepsilon = \min\cbr{\varepsilon_1 - \rho(x_1, x), \varepsilon_2 - \rho(x_2, x)}$
    Проверим условие $O_{\varepsilon} \subset O_{\varepsilon_1}(x_1) \cap O_{\varepsilon_w}(x_2)$
    % todo: вставить картинку

    Проверим $y \in O_{\varepsilon_1}(x_1) \Rightarrow \rho(y, x_1) < \varepsilon_1$.
    \[
        \rho(y, x_1) \leq \rho(y, x) + \rho(x, x_1) < \varepsilon - \rho(x, x_1) + \rho(x, x_1) = \varepsilon
    \]
    Аналогично для второго шара.

    Т.о. по достаточному условию на базу открытые шары будет образовывать базу.
\end{proof}

Простой способ сравнения топологий. Пусть $X$ - множество. $\tau_1, \tau_2$ - топологии, определенные на $X$.
ЧУМ - частично упорядоченное множество.
\begin{definition}
    $\tau_1 \leq \tau_2 \Leftrightarrow \tau_1 \subset \tau_2$
\end{definition}
Обычными словами: любое открытое множество в $\tau_1$ будет открытым в $\tau_2$.

\begin{example}
    Рассмотрим антидискретную и дискретную топологии. 
    \[
        \tau_1 = \cbr{\emptyset, X} \subset \tau_2 = 2^{X}
    \]
    Можно считать, что это два полюса сравнения, где слева - слабейшая, справа - сильнейшая.  

    Любую топологию можно сравнить с этими двумя.
    Но точно существуют не сравнимые.
\end{example}

\begin{exercise}
    Метризумые ли тривиальные топлогии(= антидискретная и дискретная)?

    1. можнов ввести дискретную метрику $\rho_{D}(x, y) = (1, x = y; 0, x \neq y)$. Получим дискретную топологию.

    2. неметризуемо.
\end{exercise}

\begin{definition}[индуцированная топология подространства]
    Пусть $(X, \tau)$ - топологическое пространство, $Y \subset X$. $\tau_{Y} = \cbr{U \cap Y : U \in \tau}$.
\end{definition}
\begin{proof}
    Очевидно, что выполняются аксиомы топологии.
\end{proof}


\begin{example}
    $\R^2 = X$ - метрическое пространство, $Y \subset X$. 
    % todo: вставить картинку 
\end{example}

\begin{definition}
    $U$ - окрестность точки $x \in X$ = $U \in \tau$ такое, что $x \in U$. 
\end{definition}

\begin{nota_bene}[Есть тут глубокий смысл?]
    \[
        \bigcap_{i = 1}^{n} \text{окрестность точки } x = \text{окрестность точки x}
    \]

    \[
        \bigcup_{\alpha} \text{окрестность } x = \text{окрестность}
    \]
\end{nota_bene}

\begin{statement}
    $A \subset X$ - открыто $\Leftrightarrow$ для каждой точки $x \in$ сущесвтует ее окрестность, лежащая в $A$.
\end{statement}
\begin{proof}
    % todo: вставить картинку
    
    $\br{\Leftarrow}$: Рассмоторим $C = \bigcup_{x \in A} O(x) \in \tau$.
    Очевидно, что $A \subset C$. А т.к. для каждого $x \in A$ верно $O(x) \subset A$, то также выполняется включение в другую сторону.

    $\br{\Rightarrow}$: раз $A$ - открыто, то $A$ является окрестностью.
\end{proof}

\begin{definition}
    Пусть $x \in X$, $\cbr{x} \in \tau$, то $x$ называется изолированной точкой.
\end{definition}

\begin{nota_bene}
    Если топлопгия дискретная, то все точки изолированные.
\end{nota_bene}

\begin{definition}
    Пусть $A \subset X$, $x \in X$. $x$ - точка прикосновения множества $A$, если для любой окрестности $O(x)$ выполняется $O(x) \cap A \neq \emptyset$. 
\end{definition}

\begin{definition}
    $x \in A$ - внутрення точка множества $A$, если существует $O(x)$: $O(x) \subset A$.
\end{definition}

\begin{definition}[A1]
    Замыкание множества $A$ - множество всех точек прикосновения $A$.
    Обозначается $\overline{A}$.
\end{definition}


% todo: добавить оператор \Int, дописать определения
\begin{definition}[B1]
    Внутренность - множество всех внутренних точек.
    Обозначается $Int(A)$.
\end{definition}

\begin{exercise}
    $Int(A) \subset A \subset \overline{A}$
\end{exercise}

\begin{definition}[A2]
    $\overline{A} = \bigcap_{\text{по всем возможным } F} = {F : 1. F - \text{замкнуто}}, 2. A \subset F$
    $\overline{A}$ = наименьшее замкнутое множество, содержащее $A$.

\end{definition}

\begin{definition}[B2]
    $Int(A) = \bigcup_{\text{по всем U}} U : U \in \tau, U \subset A$
    $Int(A)$ = наибольшее открытое в $A$.
\end{definition}


todo: 
\begin{definition}
    $x \in X$ - граничная точка $A$, если $x$ - точка прикосновения и $x \notin Int(A)$.  

    Граница - множество граничных точек.
    Обозначается $Bd(A)$.
\end{definition}

\begin{nota_bene}
    \[
        Bd(A) = \overline{A} \setminus Int(A)
    \]
\end{nota_bene}

\begin{theorem}
    Это определния эквивалентны.
\end{theorem}
\begin{proof}
    Докажем эквивалентность определний B1 и B2.

    Пусть $Int(A)$ - множество точек в смысле определения B1. Докажем B2.
    Докажем проверкой включений.

    $(\subseteq)$: если $x \in A$ - внутрення точка, то существует $O(x) \subset A$, тогда $x \in Int(A)$ в смыле другого определения.

    $(\supseteq)$: $x \in Int(A)$ в смысле определния B2, тогда $x$ принадлежит каком-то одному открытому $V \subset A$, тогда можем взять $V$ за окрестность точки $x$.
\end{proof}

% todo: добавить картинку
\begin{definition}[понятие непрерывного отображения]
    Пусть $f: X \rightarrow Y$. $f$ непрерывно в точке $x_0 \in X$, если для каждой $O(f(x_0))$ существует такая окрестность $O(x_0)$, что $f(O(x_0)) \subset O(f(x_0))$. 

    $f$ - непрервыное отображение топологических пространств, если оно непрервыно во всех $x \in X$.
\end{definition}

% todo: дописать доказательство
\begin{statement}
    Следующие условия эквивалентны:
    \begin{enumerate}
        \item $f$ - непрерывно
        \item прообраз любого открытого множества является открытым, т.е. $U \in \tau_{X} \Rightarrow f^{-1}(U) \in \tau_{Y}$
        \item прообраз любого замкнутого замкнут
        \item $f(\overline{A}) = \overline{f(A)}$ 
    \end{enumerate}
\end{statement}
\begin{proof}
    Докажем только $(1) \Leftrightarrow (2)$.

    $(\Rightarrow)$: пусть $f$ - непрерывно. Нужно доказать, что $f^{-1}(V)$ - открыто, можем воспользоваться утвержеднием при критерий открытости.


    $(\Leftarrow)$: пусть $x \in X$, $V$ - окрестность точки $f(x_0)$, тогда по предположению $f^{-1}(V)$ - открыто, следовательно существует $O(x) \subset f^{-1}(V)$
\end{proof}

\begin{exercise}
    Доказать остальные эквивалентности в утверждении выше.
\end{exercise}
