\section{Лекция 13}

\subsection{Фундаментальная группа}

Пусть $X$ - топологическое пространство, выберем в нем две точки $x_0, x_1$. Рассмотрим множество путей из $x_0$ в $x_1$.

\begin{definition}
    Пусть - непрерывное отображение $\gamma: [0,1] \rightarrow X$.
\end{definition}

\begin{definition}[гомотопия путей - связная гомотопия]
    Гомотопия, сохраняющее точки на граница отрезка, для отображений - $\gamma_0, \gamma_1: [0,1] \rightarrow X, \gamma_i(0) = x_0, \gamma_i(1) = x_1$.
\end{definition}

\begin{definition}
    Пространство путей из $x_0$ в $x_1$ = $\cbr{\gamma : [0,1] \rightarrow X : \ \gamma(0) = x_0, \gamma(1) = x_1}$. Обозначается $\Omega(x_0, x_1)$.
\end{definition}

\begin{nota_bene}
    Отношение гомотопности - отношение эквивалентности.
\end{nota_bene}

\begin{definition}
    $\pi^{x}(x_0, x_1) = \faktor{\Omega(x_0, x_1)}{\sim}$ - множество гомотопических классов путей
\end{definition}

Умножение путей. Можем умножать пути, у которых начало первого и нонец второго совпадают, т.е $\gamma_1(1) = \gamma_2(0)$, где $\gamma_1$ - путь от $x_0$ до $x_1$, $\gamma_2$ - путь от $x_1$ до $x_2$.
\begin{align}
    \gamma_1 \gamma_2 = \begin{cases}
        \gamma_1 (2t), & 0 \leq t \leq \frac{1}{2} \\
        \gamma_2 (2t - 1), & \frac{1}{2} < t \leq 1
    \end{cases}
\end{align}

Видно, что операция не ассоциативна, но $\gamma_1 (\gamma_2 \gamma_3) \sim (\gamma_1 \gamma_2) \gamma_3$. Докажем это.
\begin{proof}
    Построим гомотопию.
    \begin{align}
        H(t, s) = \begin{cases}
            \gamma_1(\frac{4}{s + 1} t) & 0   \leq t < t_1 = \frac{s + 1}{4} \\
            \gamma_2(4t - 4 t_1) & t_1 \leq t < t_2 = \frac{s + 2}{4} \\ 
            \gamma_3(???) & t_2 \leq t \leq 1 \\ 
        \end{cases}
    \end{align}
\end{proof}

Таким образом умножение путей ассоциативно с точностью до гомотопии.

\begin{definition}
    Петля с фиксированный началом = замкнутый путь, т.е. $\gamma(0) = \gamma(1) = x_0 \in X$.
\end{definition}

\begin{definition}
    Рассмотрим $\faktor{\Omega(x_0, x_0)}{\sim}$ с операцией умножения путей. Это фундаментальная группа, обозначается $\pi_1(X, x_0)$
\end{definition}

\begin{nota_bene}
    Умножение классов эквивалентности $[\gamma_1][\gamma_2] = [\gamma_1 \gamma_2]$.
\end{nota_bene}

\begin{theorem}
    Это действительно группа.
\end{theorem}

\begin{proof}
    Проверим корректность. Пусть $\gamma_i, \gamma'_i \in [\gamma_i]$, необходимо доказать, что $[\gamma_1][\gamma_2] = [\gamma'_1] [\gamma'_2]$ - очевидно.
    
    Ассоциативность уже доказана.

    Существование нейтрального элемента $e = [\gamma: [0,1] \rightarrow x_0]$

    Обратный элемент $[\gamma]^{-1} = [\gamma(1 - t)]$
\end{proof}

Зависимость $\pi_1$ от начальной точки. Выберем две различные точки в пространстве $X$ - $x_0$ и $x_1$. И между этими точками существует путь $\tilde{\gamma}$.

Петле(классу гомотопической эквивалентности) из $[\gamma ]\in \faktor{\Omega(x_0)}{\sim}$ сопоставляем путь $[\tilde{\gamma} \gamma \tilde{\gamma}^{-1}]$, т.е. мы задали отображнение из $\pi_1(X, x_1)$ в $\pi_1(X, x_2)$, обозначим его $[\gamma]^*$.

\begin{statement}
    $[\gamma]^*$ изоморфизм групп
\end{statement}
\begin{proof}
    Очевидно.
\end{proof}
Таким образом, если $X$ - линейно связно, то фундаментальная группа не зависит от выбора точки.

\begin{definition}
    Если фундаментальная группа тривиально, то пространство называется односвязным.
\end{definition}


 
