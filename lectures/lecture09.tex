\section{Лекция 9}

\begin{exercise}
    Непрерывное отображение (гомеоморфизм) треугольника на квадрат.
    $f: \Delta \rightarrow I \times I$

    \[
        f(x, y) =
        \begin{cases}
            (x + y, x + y), & x > y \\
            (2x, 2x), & x = y \\
            (x + y, x + y), & x < y
        \end{cases} = (x + \min{(x, y)}, x + \min{(x, y)})
    \]
    % \begin{center}
    %     \begin{tikzpicture}
    %         \draw (0,0) -- (1,0) -- (0, 1) -- (0,0);
    %         \draw (0,0) -- (0.5, 0.5);
    %         \draw (0,0.5) -- (0.5, 0.5);
    %         \draw (0.5,0) -- (0.5, 0.5);
    %     \end{tikzpicture}
    % \end{center}
\end{exercise}

\subsection{Теорема Титца о продолжении непрерывной функции}

\begin{theorem}[Титца о продолжении непрерывной функции]
    Пусть $X$ - нормальное топологическое пространство. $F \subset X$ - замкнутое подмножество. $\varphi: F \rightarrow \R$ - непрерывная ограниченная(т.е. $\norm{\varphi} = \sup_{x \in F}|\varphi(x)| < \infty$) функция.

    Тогда существует $\Phi: X \rightarrow \R$ - непрерывное продолжение функции $\varphi$, которое сохраняет норму $\norm{\Phi} = \sup_{y \in X} \abs{\Phi(y)} = \norm{\varphi}$

\end{theorem}
\begin{proof}
    Будем строить две последовательности функций.

    \begin{enumerate}
        \item $\Phi_n: X \rightarrow \R$
        \item $\varphi_n: F \rightarrow \R$
    \end{enumerate}

    \begin{nota_bene}
        Пусть $f_n(x) : Y \rightarrow \R$ - фундаментальная последовательность, тогда существует $f(y) = \lim_{n \to \infty} f_n(y)$
        \begin{exercise}
            Доказать, что фундаментальная последовательность равномерно сходится.
        \end{exercise}
    \end{nota_bene}

    Алгоритм построения последовательностей.

    \begin{enumerate}
        \item $\varphi_0 = \varphi$, так как $\varphi$ - ограниченная функция, то выполняется $\norm{\varphi} = \norm{\varphi_0} = M_0 < +\infty$
        Определим два замкнутых множества в $X$
        \[
            A_0 = \cbr{x \in F : \varphi(x) = \varphi_0(x) \leq -\frac{M_0}{3}}
        \]
        \[
            B_0 = \cbr{x \in F : \varphi(x) = \varphi_0(x) \geq \frac{M_0}{3}}
        \]
        Очевидно, что эти множества являются замкнутыми и непересек.

        Применим лемму Урысона к отрезку $\sqbr{-\frac{M_0}{3}, \frac{M_0}{3}}$, получим функцию $\Phi_0(x)$, которая на $A_0$ тождественна $-\frac{M_0}{3}$, на $B_0$ тождественна $\frac{M_0}{3}$.

        Рассмотрим "номру" функции $\Phi_0$: $\norm{\Phi_0} \leq \frac{M_0}{3}$.
        \item определим функции $\varphi_1 = \varphi_0 - \Phi_0$ на множестве $F$. Эта функция непрерывна. Рассмотрим норму введеной функции на 3-ех участках.
        \begin{enumerate}
            \item $\varphi_0 \geq \frac{M_0}{3}$; $\Phi_0 = \frac{M_0}{3}$
            \item $ - \frac{M_0}{3} \leq \varphi_0 \leq \frac{M_0}{3}$; $-\frac{M_0}{3} \leq \Phi_0 \leq \frac{M_0}{3}$
            \item $\varphi_0 \leq -\frac{M_0}{3}$; $\Phi_0 = -\frac{M_0}{3}$
        \end{enumerate}
        Из этих неравенств видно, что
        \[
            \norm{\varphi_1} \leq \frac{2 M_0}{3}
        \]
        % \begin{statement}
        %     Норму этой функции можно ограничить числом < 1. 
        % \end{statement}
        % todo: добавить картинку
        \item Примени тоже построение, что и в предыдущем пункте, счетное число раз и получим две последовательности функций. 
    \end{enumerate}

    Таким образом получили, что
    \begin{center}
        $\varphi_{n + 1} = \varphi_n - \Phi_n$ на множестве $F$
    \end{center}
    И
    \begin{align}
        & \norm{\Phi_{n + 1}} \leq \frac{M_0}{3}
        & \norm{\varphi_{n + 1}} \leq \frac{2 M_0}{3}
    \end{align}
    Следовательно
    \begin{align}
        & \norm{\Phi_{n + 1}} \leq \frac{1}{3} \br{\frac{2}{3}}^{n - 1} M_0
        & \norm{\varphi_{n + 1}} \leq \br{\frac{2}{3}}^n M_0
    \end{align}

    Рассмотрим ряд $\sum_{i = 0}^{\infty}\Phi_i$. Докажем, что последовательность частичных сумм $S_n = \sum_{i = 0}^{n}\Phi_i$ будет фундаментольной.
    \[
        \norm{S_n - S_m} = \norm{S_{m + 1} + \ldots + S_n} \leq \sum_{i = 0}^{\infty}\abs{\Phi_i} \leq \frac{M_0}{3} \br{\frac{2}{3}}^{n + 1} \sum_{l = 0}^{\infty} \br{\frac{2}{3}}^l = M_0 \br{\frac{2}{3}}^{n + 1} < \varepsilon
    \]
    Таким образом она сходится и $S_n \rightrightarrows \Phi$

    Докажем, что $\Phi$ совпадает с $\varphi$ на $F$.
    \[
        \norm{\varphi - \sum_{i = 0}^{n} \Phi_i} = \norm{\varphi_{n + 1}} \leq \br{\frac{2}{3}}^n M_0
    \]
    Припредельном переходе получим $\varphi = \Phi$ на множестве $F$.
\end{proof}
