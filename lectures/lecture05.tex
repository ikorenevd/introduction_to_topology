\section{Лекция 5}


Рассмотрим полезную характеристику $T_1$-пространства: 

\begin{statement}
    $X$ является $T_1$-пространством тогда и только тогда, когда для любого $x \in X$ множество $\{x\}$ замкнуто.      
\end{statement}
\begin{proof}
    $(\Rightarrow):\ $ Пусть $X$ является $T_1$-пространством. Если мы возьмём $y \neq x$, то существуют их окрестности $O(x)$ и $O(y)$, т.ч. $y \not\in O(x)$ и $x \not\in O(y) \Longrightarrow y$ не является точкой прикосновения множества $\cbr{x}$. Значит, в $X \setminus \{x\}$ нет точек прикосновения множества $\cbr{x}$. Таким образом, $x$ --- единственная точка прикосновения множества $\cbr{x}$, значит, это множество замкнуто.

    $(\Leftarrow):\ $ Пусть $x \neq y$ и множества $\{x\}$ и $\{y\}$ замкнуты, тогда $X \setminus \{x\}$ и $Y \setminus \{y\}$ открыты, а значит, мы можем взять их в качестве окрестностей: $y \in X \setminus \{x\}$,  $x \in X \setminus \{y\}$. Получаем, что $X$ является $T_1$-пространством. 
\end{proof}

\begin{nota_bene}[Примечание редактора]
    Определения точки прикосновения и предельной точки подмножества $A$ в топологическом пространстве $X$, вообще говоря, отличаются, причём в разной литературе используются разные определения.
    В книгах Виро О.Я., Иванова О.А., Нецветаева Н.Ю., Харламова В.М; Федорчука В.В., Филиппова В.В и в статье на Википедии определение точки прикосновения совпадает с данным в этом курсе. Однако:
    В книге Виро О.Я., Иванова О.А., Нецветаева Н.Ю., Харламова В.М. и в статье на Википедии предельная точка подмножества $A$ в топологическом пространстве $X$ определяется как точка, любая проколотая окрестность которой имеет непустое пересечение с $A$,
    а в книге Федорчука В.В., Филиппова В.В предельная точка подмножества $A$ в топологическом пространстве $X$ определяется как точка, в любой окрестности которой содержится бесконечно много точек из $A$. В книге Виро О.Я., Иванова О.А., Нецветаева Н.Ю., Харламова В.М. и в статье на Википедии точки с таким свойством называются точками накопления множества $A$.
    Предельные точки, точки прикосновения и точки накопления образуют, вообще говоря, разные множества для данного $A \subset X$. Множество точек прикосновения распадается на множество предельных точек и множество изолированных точек. Если $X$ является $T_1$-пространством, то понятия предельной точки и точки накопления равносильны.

    Обо всём этом лектор решил не говорить, и употреблял термины "предельная точка" \, и "точка прикосновения" \,, по-видимому, как равносильные.
\end{nota_bene}

\begin{statement}
    $T_2 \Rightarrow T_1 \Rightarrow T_0$, но, вообще говоря, из $T_3$ не следует $T_0$ (а значит, не следуют и $T_1$, и $T_2$).
\end{statement}
\begin{proof}
    Импликации очевидны. Приведём контрпример для $T_3$-пространства: пусть $X = \{x, y\}$ и $\mcT = \{\es, X\}$. Рассмотрим точку $x$, тогда любое замкнутое подмножество $F \subset X$, не содержащее $x$, является пустым. Берём в качестве окрестностей: $O(F) = \es, \ O(x) = X$. Тогда $X$ является $T_3$-пространством, но не является $T_0$-пространством.  
\end{proof}

\begin{definition}
    Пространство $X$ называется регулярным, если оно удовлетворяет аксиомам отделимости $T_3$ и $T_0$.   
\end{definition}

\begin{nota_bene}[Примечание редактора]
    В разной литературе регулярные пространства определяются или как $T_3, T_0$-пространства, или как $T_3,T_1$-пространства. Эти определения равносильны, что моментально следует из следующего утверждения.
\end{nota_bene}

\begin{statement}
    $T_3$ и $T_0 \Rightarrow T_2$.
\end{statement}
\begin{proof}
    Возьмём $x$ и $y$, такие что $x \neq y$. Без ограничения общности: по $T_0$ существует $O(x)$: $y \not\in O(x)$. Рассмотрим множество $F = X \setminus O(x)$. $F$ замкнуто и $y \in F$. По аксиоме $T_3$ существуют окрестности $\widetilde{O}(x)$ и $O(F)$: $\widetilde{O}(x) \cap O(F) = \es$. При этом существует такая окрестность $O(y)$ точки $y$, что $O(y) \subset O(F)$, так как $y \in O(F)$ и $O(F)$ открыто. Таким образом, $\widetilde{O}(x) \cap O(y) = \es$, значит, $X$ является $T_2$-пространством.
\end{proof}

\begin{example}
    Если $X$ --- метрическое пространство, то $X$ хаусдорфово.
\end{example}

$\newline$ 
Рассмотрим полезную характеристику $T_2$-пространства:
\begin{statement}
    $X$ --- $T_2$-пространство $\Leftrightarrow \forall x \in X: \ \bigcap \overline{O}(x) = \{x\}$, где пересечение берётся по всем окрестностям точки $x$.  
\end{statement}
\begin{proof}
    $\Rightarrow$ Очевидно, $x \in \bigcap \overline{O}(x)$, где пересечение берётся по всем окрестностям точки $x$. Докажем методом от противного: пусть существует $y \in \bigcap \overline{O}(x)$. Тогда $\forall \, \overline{O}(x): \ y \in \overline{O}(x)$. Т.к. Так как $X$ является $T_2$-пространством, то существуют окрестности $U(x)$ и $U(y)$: $U(x) \cap U(y) = \es$. Но $y \in \overline{O}(x) \Leftrightarrow \forall V(y) \ \forall O(x): \ V(y) \cap O(x) \neq \es$. Получаем противоречие, значит, $\cbr{x} = \bigcap \overline{O}(x)$. \\
    $\Leftarrow$ Доказательство остаётся в качестве упражнения читателю. 
\end{proof}

\begin{statement}
    Из $T_4$ не следует ни $T_3$, ни $T_0$ (а значит, не следуют и $T_1$, и $T_2$). 
\end{statement}
\begin{proof}
    Приведём оба контрпримера: \\
    1) Пусть $X = \R$, $\mcT = \{\{(a, +\infty),\ a \in \R\}, \es, \R\}$. Замкнутые множества имеют вид $F = (-\infty, a]$, или $X$, или $\es$. Так как в $X$ нет двух непустых замкнутых непересекающихся множеств, то $X$ является $T_4$-пространством. Возьмем закнутое множество $F = (-\infty, a]$ и точку $b \not\in F$. Единственной окрестностью $F$ является всё пространство $X = \R$, а значит, любая окрестность точки $b$ будет иметь непустое пересечение с любой окрестностью $F$. Получаем, что $X$ не является $T_3$-пространством. \\
    2) Рассмотрим связное двоеточие: $X = \{x, y\}$ и $\mcT = \{\es, X\}$. Замкнутых множеств всего два $\{\es, X\}$. Можем взять $F_1 = \es$, $F_2 = X$ или $F_1 = \es, F_2 = \es$ --- очевидно, $X$ является $T_4$-пространством, но не является $T_0$-пространством.
\end{proof}

\begin{statement}
    $T_4 + T_1 \Rightarrow T_3$.
\end{statement}
\begin{proof}
    Т.к. $X$ --- $T_1$-пространство, то: $\forall x \in X:$ множество $\{x\}$ замкнуто. Пусть $F_1 = \{x\}$, $F$ --- любое другое замкнутое подмножество в $X$, не содержащее $x$. По $T_4$: существуют непересекающиеся окрестности $O(\cbr{x})$ и $O(F)$ этих множеств. Но $O(\cbr{x}) = O(x)$, а значит, $X$ является $T_3$-пространством.
\end{proof}

\begin{definition}
    Пространство $X$ называется нормальным, если оно удовлетворяет аксиомам отделимости $T_4$ и $T_1$.
\end{definition}

\begin{lemma}
    Пусть в метрическом пространстве $(X, \rho)$ множества $F_1, F_2$ замкнуты и не пересекаются. Тогда $\forall x \in F_1$ $\exists \veps > 0:\ O_{\veps}(x)\cap F_2 = \es$.
\end{lemma}
\begin{proof}
    Предположим противное: пусть такую окрестность найти нельзя, то есть $\forall \veps > 0: \ O_{\veps}(x) \cap F_2 \neq \es$. Тогда $x \in \overline{F_2}$, но $\overline{F_2} = F_2 \Longrightarrow x \in F_1 \cap F_2 \Longrightarrow F_1 \cap F_2 \neq \es$ --- противоречие.  
\end{proof}

\begin{theorem}
    Метрическое пространство нормально.
\end{theorem}
\begin{proof}
    Метрическое пространство хаусдорфово, то есть выполняется аксиома $T_2$, из которой следует аксиома $T_1$. Докажем $T_4$. Пусть $F_1, F_2$ -- замкнутые непересекающиеся множества. Возьмём точку $x_1 \in F_1$ и рассмотрим окрестность $O_{\veps(x_1)}(x_1): O_{\veps(x_1)}(x_1) \cap F_2 = \es$. Аналогично для $x_2 \in F_2$. Тогда рассмотрим окрестности
    \[
        V(F_1) = \underset{x_1 \in F_1}{\bigcup} O_{\frac{\veps(x_1)}{2}}(x_1) \text{ и } W(F_2) = \underset{x_2 \in F_2}{\bigcup} O_{\frac{\veps(x_2)}{2}}(x_2).
    \]
    Докажем, что $V(F_1) \cap W(F_2) = \es$. Предположим противное: $\exists w \in V(F_1) \cap W(F_2)$. Тогда $\exists x_1 \in F_1: w \in O_{\frac{\veps(x_1)}{2}}(x_1)$ и $\exists x_2 \in F_2: w \in O_{\frac{\veps(x_2)}{2}}(x_2)$. Заметим, что $\rho(x_1, w) < \frac{\veps(x_1)}{2}$ и $\rho(x_2, w) < \frac{\veps(x_2)}{2}$, тогда по неравенству треугольника $\rho(x_1,x_2) < \max\cbr{\veps(x_1), \veps(x_2)}$. Без ограничения общности: $\veps(x_2) > \veps(x_1)$. Тогда $x_1 \in O_{\veps(x_2)}(x_2)$, а также $x_1 \in F_1$, но $O_{\veps(x_2)}(x_2)$ построена так, что она не пересекается с $F_1$ --- противоречие. Значит, метрические пространства являются $T_1, T_4$-пространствами, т.е. являются нормальными.
\end{proof}