\section{Лекция 3}

% todo: заменить объединение на дизъюнктное объединение

%bigchange! Убрал доказательство утверждения из 2 лекции: переместил его туда 

\begin{nota_bene}[филосовское]
    Проверять непрерывность $f: X \rightarrow Y$ удобно проверять на уровне базы или предбазы.
    Пусть $\beta \subset 2^Y$ - база топологии $Y$.
    Прообраз базы (предбазы) открыт: $f^{-1}(\beta) \subset \tau_X$
\end{nota_bene}

\begin{example}
    \begin{enumerate}
        \item $f: \R \rightarrow \R$ - непрерывные функции одной переменной из математического анализа
        \item $f(x) = e^{2 \pi i x} = \cos (2\pi x) + i \sin (2\pi x)$, $f: \R^1 \rightarrow S^1$ - общее название таких отображений (накрытие)
% todo: добавить картинку "спирали" 
        \item тривиальный пример - постоянное отображение. $f(x) = y_0$, где $f: X \rightarrow Y$ и $y_0 \in Y$.
        \item композиция непрерывных - непрерывное отображение $X \rightarrow_{f} Y \rightarrow_{g} Z$, тогда $g(f)$ - непрерывно.
        \item $Z \subset X \rightarrow_f Y$, где на $Z$ индуцирована топология $X$. $i: Z \rightarrow X$, $i(x) = x$, тогда $i$ непрерывно в индуцированной топологии.
    \end{enumerate}
\end{example}

Непрерывность в метрических пространствах.

\begin{definition}[По Коши]
    $f$ непрерывно в точке $x_0$, если для каждого $\varepsilon > 0$ существует $\delta > 0$ такое, что $f(O_{\delta}(x_0)) \subset O_{\varepsilon}(f(x_0))$
\end{definition}
\begin{definition}[По Гейне]
    ...
\end{definition}

\begin{exercise}
    Доказать эквивалентность определений.
\end{exercise}

\begin{theorem}[Кривая Пеано]
    Существует непрерывное отображение $f: [0, 1] \rightarrow [0, 1] \times [0, 1]$
\end{theorem}
Докажем эту теорему потом.

\begin{definition}
    $f: X \rightarrow Y$ - гомеоморфзим, если $f$ - биекция и $f, f^{-1}$ - непрерывны.

    Если сущесвтует гомеоморфзим между $X$ и $Y$, то эти пространства гомеомрфны.
\end{definition}

\begin{nota_bene}
    Гомеомрфзим задает отношение эквивалентности.

    Чтобы доказать, что пространства не являются гомеоморфными, нужно найти свойства пространств, которые должные сохраняться, но у этих пространств они отличаются. 
\end{nota_bene}

\begin{example}
    $f(x) = tg(x) : (-\frac{\pi}{2}, \frac{\pi}{2}) \rightarrow_f (-\infty, +\infty) = \R$ - гомеоморфзим.  
\end{example}

Связность и линейная связность.

\begin{definition}
    Пространство $X$ называется несвязным, если его можно представить в виде объединения двух непересекающихся непустых открытых подмножетсв.

    $X$ - связно, если нельзя так разбить.
\end{definition}

\begin{example}
    \begin{enumerate}
        \item Рассмотрим $X$ с дискретной топологией. $X$ - всегда несвязно, если содержит более двух точек.
        \item Рассмотрим $X$ с антидискретной топологией. $X$ - всегда связно.
        \item 
    \end{enumerate}
\end{example}

\begin{theorem}
    Отрезок $I = [0, 1]$ с индуцированной топологией - связен.
\end{theorem}
\begin{proof}
    От противного. Используя теорию действительных чисел. 
    % \[
    %     [0,1] = A \cup B
    % \]
    % Пусть $0 \in A$. Т.к. $A$ открыто, то $0$ лежит в $A$ с некоторой своей окрестностью $[0, \varepsilon)$.
    % Множество таких окрестностей ограничено, следовательно существует $\sup \varepsilon = \varepsilon_0$.

    % Рассмторим $[0, \varepsilon_0)$. Докажем, что этот интервал лежит в $A$.
    % Из свойства супремума известно, что $\varepsilon_0$ - точка прикосновения множества $[0, \varepsilon_0)$. Тогда 
\end{proof}

\begin{statement}
    Непрерывный образ связного множества - связен, т.е.
    если $X$ - связен, $f: X \rightarrow Y$ следовательно $f(X)$ - связно
\end{statement}
\begin{proof}
    От противного.
    Пусть выполняется $f(X) = A \cup B$, тогда $X = f^{-1}(A) \cup f^{-1}(B)$ - противоречие.
\end{proof}

\begin{definition}
    Пусть в топологическом пространстве, содединяющий $x_0, y_0 \in X$, это непрерывное отображение $\gamma: [0,1] \rightarrow X$ такое, что $gamma(0) = x_0$, $gamma(1) = y_0$. 
\end{definition}

\begin{nota_bene}
    $\gamma([0, 1])$ - связно.
\end{nota_bene}

\begin{definition}
    Пространство $X$ является линейно связным, если для каждый двух точек, существует путь, содединяющий их.
\end{definition}

\begin{theorem}
    Пусть $X$ линейно свзяно, тогда $X$ - связно.
\end{theorem}
\begin{proof}
    От противного. Пусть $X = A \cup B$. Тогда $x_0 \in A$, $y_0 \in B$ можно свзять отображением $\gamma$, тогда получим, что $\gamma([0, 1]) = \gamma([0, 1]) \cap A \cup \gamma([0, 1]) \cap B$ - противоречие.
\end{proof}

\begin{nota_bene}
    Обратное неверно. Пример - график $f(x) = \sin \frac{1}{x}$ с добавлением отрезка $[-1, 1]$.
\end{nota_bene}