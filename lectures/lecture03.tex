\section{Лекция 3}

%bigchange! Убрал доказательство утверждения из 2 лекции: переместил его туда 

\begin{nota_bene}\label{NB:Sufficient_condition_for_continuity_in_terms_of_base}
    Проверять непрерывность отображения топологических пространств удобно на уровне базы или предбазы:
    Пусть $\mfB \subset \mcT_{Y}$ --- база (или предбаза) топологии на $Y$.
    Тогда отображение $f: X \rightarrow Y$ непрерывно $\Longleftrightarrow$ прообраз любого элемента базы (предбазы) открыт: $f^{-1}(U) \subset \mcT_{X}, \ U \in \mfB$.
\end{nota_bene}

\begin{example} % todo: Ждём, даст ли Миллионщиков определение понятию накрытия. Сделать ссылку на определение
    \begin{enumerate}
        \item $f: \R \rightarrow \R$ --- непрерывные функции одной переменной ("функции из математического анализа").
        \item $f(x) = e^{2 \pi i x} = \cos (2\pi x) + i \sin (2\pi x)$ (Эта функция представляет собой пример накрытия $f: \R^1 \rightarrow S^1$. Определение накрытия смотри (возможно) дальше в курсе). % на стр.~(\pageref{D:}).
% todo: добавить картинку "спирали" 
        \item Тривиальный пример: постоянное отображение $f(x) \equiv  y_0$, где $f: X \rightarrow Y$ и $y_0 \in Y$.
        \item Композиция непрерывных отображений является непрерывным отображением: Пусть $X \overset{f}{\rightarrow} Y \overset{g}{\rightarrow} Z$; $f, g$ --- непрерывные отображения. Тогда $g \circ f$ --- непрерывное отображение.
        \item Пусть $(X, \mcT)$ --- топологическое пространство, $Z \subset X$, на $Z$ индуцирована топология $\mcT_{Z} = \mcT \!\! \mid_{Z}$. Рассмотрим отображение включения: $i: Z \rightarrow X$, $i(x) = x$. Тогда $i$ непрерывно в индуцированной топологии $\mcT_{Z}$.
        \item Пусть в дополнение к предыдущему пункту: существует $f: X \rightarrow Y$ --- непрерывное отображение. Рассмотрим отображение ограничения: $f \! \mid_{Z} \, : Z \rightarrow Y$. Оно непрерывно, т.к. является композицией непрерывных отображений: $f \! \mid_{Z} \ = f \circ i$.
        \item Непрерывность в метрических пространствах:
    \end{enumerate}
\end{example}

\begin{definition}[Непрерывности отображения метрических пространств по Коши]
    Пусть $(X, \rho_{X}), \ (Y, \rho_{Y})$ --- метрические пространства, $f: X \rightarrow Y$. Тогда отображение $f$ называется непрерывным в точке $x_0 \in X$,
    если $\forall \veps > 0 \ \exists \, \delta > 0 \, : f(O_{\delta}(x_0)) \subset O_{\veps}(f(x_0))$, где $O_{\delta}$ и $O_{\veps}$ --- открытые шары в пространствах $X$ и $Y$ соответственно. 

    Отображение $f$ называется непрерывным (непрерывным отображением метрических пространств), если оно непрерывно во всех $x \in X$.
\end{definition}
\begin{definition}[Непрерывности отображения метрических пространств по Гейне]
    Пусть $(X, \rho_{X}), \ (Y, \rho_{Y})$ --- метрические пространства, $f: X \rightarrow Y$. Тогда отображение $f$ называется непрерывным в точке $x_0 \in X$,
    если для любой последовательности $\{x_n\}_{n = 1}^{\infty}$ элементов $X$, сходящейся к $x_0$, последовательность $\{f(x_n)\}_{n = 1}^{\infty}$ сходится к $f(x_0)$.

    Отображение $f$ называется непрерывным (непрерывным отображением метрических пространств), если оно непрерывно во всех $x \in X$.
\end{definition}

\begin{exercise}
    Доказать эквивалентность определений непрерывности отображения метрических пространств по Коши и по Гейне.
\end{exercise}

\begin{theorem}[Кривая Пеано]
    Существует непрерывное отображение $f: [0, 1] \rightarrow [0, 1] \times [0, 1]$.
\end{theorem}
% todo: Сделать ссылку на теорему
Доказательство этой теоремы смотри дальше в курсе. %\ref{T:})

\begin{definition}
    Пусть $X, \ Y$ --- топологические пространства, $f: X \rightarrow Y$. Тогда отображение $f$ называется гомеоморфизмом, если: \quad
    1) $f$ --- биекция, \quad 2) $f$ непрерывно, \quad 3) $f^{-1}$ непрерывно.

    Если между пространствами $X$ и $Y$ существует гомеоморфизм, то эти пространства называются гомеоморфными.
\end{definition}

\begin{nota_bene}
    Свойство "быть гомеоморфными" \, очевидно является отношением эквивалентности на множестве топологических пространств, а значит, разбивает это множество на классы эквивалентности.

    Чтобы доказать, что пространства не являются гомеоморфными, можно найти топологические свойства этих пространств, которые должны сохраняться при любом гомеоморфизме, но у этих пространств отличаются. 
\end{nota_bene}

\begin{example}
    $f(x) = \tg(x) : \br{-\frac{\pi}{2}, \frac{\pi}{2}} \overset{f}{\rightarrow} \br{-\infty, +\infty} = \R$ --- гомеоморфизм.  
\end{example}

\subsection{Связность и линейная связность.}

\begin{definition}
    Топологическое пространство $X$ называется несвязным, если его можно представить в виде объединения двух непустых непересекающихся открытых подмножеств.

    Если же пространство $X$ так разбить нельзя, то оно называется связным.
\end{definition}

\begin{example} %bigchange! Заменил требование на "более двух элементов" на "более одного элемента", простое двоеточие ведь очевидно несвязно
    \begin{enumerate}
        \item Любое пространство с дискретной топологией несвязно, если содержит более одного элемента.
        \item Любое пространство с антидискретной топологией связно.
    \end{enumerate}
\end{example}

\begin{theorem} %bigchange! Дописал и формализовал доказательство
    Отрезок $I = [0, 1]$ с топологией, индуцированной естественной топологией вещественной прямой (т.е. топологией, порождённой на $\R$ евклидовой метрикой), связен.
\end{theorem}
\begin{proof}
    Заметим, что в условиях теоремы открытыми подмножествами отрезка $I$ считаются интервалы вида $(a, b)$, где $0 < a < b < 1$; полуинтервалы вида $\lseg{0, a}$, где $0 < a \leq 1$; полуинтервалы вида $\rseg{b, 1}$, где $0 \leq b < 1$; сам отрезок $I$ и $\es$; а также их всевозможные объединения и конечные пересечения.

    Докажем теперь теорему от противного: пусть отрезок $I$ связен, т.е. $I = A \cup B$, \ где: \ 1) $A, B \neq \es$; \ 2) $A \cap B = \es$; \ 3) $A, B$ --- открыты.
    Без ограничения общности можем считать, что $0 \in A$. Т.к. $A$ открыто, то $0$ лежит в $A$ вместе с некоторой своей окрестностью. Тогда или эта окрестность нуля совпадает со всем отрезком: $I \subseteq A  \ \Rightarrow \ I = A \ \Rightarrow \ B = \es$ --- получаем противоречие, или эта окрестность нуля представляет собой полуинтервал, т.е. $\exists \, \veps, \, 0 < \veps \leq 1 : \ \lseg{0, \veps} \subseteq A$.
    Множество таких $\veps$ ограниченно ($0 < \veps  \leq 1$), следовательно, существует его супремум. Обозначим это множество $\Omega \ (\Omega \subseteq A)$, а его супремум --- $\veps_0$:
    \[
        \sup_{\veps \in \rseg{0, 1}}{\Omega} = \sup_{\veps \in \rseg{0, 1}}{ \cbr{\, \veps \, \mid \, \lseg{0, \veps} \subseteq A \,} } = \veps_0.
    \]
    Докажем теперь, что тогда $\sqbr{0, \veps_0} \subseteq A$. Т.к. $\veps_0$ --- супремум множества $\Omega$, то по одному из свойств супремума: \\
    $\forall \delta > 0 \ \exists \, \veps > 0: \ \veps_0 - \delta < \veps < \veps_0 \ \Rightarrow \ \veps \in \Omega$, \, т.е. \, $\lseg{0, \veps} \subset \Omega \subset A$. Значит, $\veps_0$ является точкой прикосновения множества $\Omega$, а значит, и точкой прикосновения множества $A$.
    
    Т.к. $A$ и $B$ являются открытыми и дополняют друг друга до $I$, то в индуцированной топологии на $I$ они являются одновременно открытыми и замкнутыми. Значит, $A = \overline{A}$, т.е. $A$ содержит все свои точки прикосновения. Значит, $\veps_0 \in A$. Но т.к. $A$ открыто, то $\exists \, U(\veps_0)$ --- окрестность $\veps_0$: \ $U(\veps_0) \subset A$.
    Но тогда или $\veps_0 = 1$, а значит, $A = I \ \Rightarrow \ B = \es$ --- противоречие, или $\veps_0 \neq 1 \ \Rightarrow \ \exists \, \delta > 0: \ \lseg{0, \veps_0 + \delta} = \sqbr{0, \veps_0} \cup \lseg{\veps_0, \veps_0 + \delta} \subset A \ \Rightarrow \ \veps_0 + \delta \, \in \, \Omega$ --- противоречие с тем, что $\veps_0 = \sup_{\veps \in \rseg{0, 1}}{\Omega}$.

    Во всех случаях получаем противоречия, значит, исходное предположение неверно, а значит, отрезок $I$ в индуцированной топологии связен.
\end{proof}

\begin{statement}
    Непрерывный образ связного пространства связен, т.е. если $X$ связно, $f: X \rightarrow Y$ --- непрерывное отображение, то $f(X)$ связно.
\end{statement}
\begin{proof}
    От противного: пусть $f(X) = A \cup B$, \, где \, 1) $A, B \neq \es$; \ 2) $A \cap B = \es$; \ 3) $A, B$ --- открыты в индуцированной c $Y$ на $f(X)$ топологии. Но тогда $X = f^{-1}(A) \cup f^{-1}(B)$, причём \,
    1) $f^{-1}(A), f^{-1}(B) \neq \es$, т.к. $A, B \neq \es$; \ 2) $f^{-1}(A) \cap f^{-1}(B) = \es$, т.к. $A \cap B = \es$; \ 3) $f^{-1}(A), f^{-1}(B)$ --- открыты в $X$, т.к. $A, B$ --- открыты в индуцированной с $Y$ на $f(X)$ топологии, а $f$ --- непрерывное отображение.
    Значит, $X$ несвязно --- противоречие.
\end{proof}

\begin{definition}
    Путь $\gamma$ в топологическом пространстве $X$, соединяющий точки $x_0, y_0 \in X$ --- это непрерывное отображение $\gamma: [0,1] \rightarrow X$ такое, что $\gamma(0) = x_0$, $\gamma(1) = y_0$.
    Точка $x_0$ называется началом пути $\gamma$, а точка $y_0$ --- концом пути $\gamma$.
\end{definition}

\begin{nota_bene}
    Из доказанных теоремы и утверждения следует, что $\gamma([0, 1])$ --- связно в топологии, индуцированной с области значений.
\end{nota_bene}

\begin{definition}
    Пространство $X$ называется линейно связным, если для любых двух точек $x, y \in X$ существует путь $\gamma$, соединяющий их и лежащий в пространстве $X$, т.е. $\gamma(\sqbr{0, 1}) \subset X$.
\end{definition}

\begin{theorem}
    Пусть $X$ линейно связно. Тогда $X$ связно.
\end{theorem}
\begin{proof}
    От противного: Пусть $X = A \cup B$, где \, 1) $A, B \neq \es$; \ 2) $A \cap B = \es$; \ 3) $A, B$ --- открыты в топологии на $X$. Т.к. $X$ линейно связно, то $\forall x_0 \in A$ и $\forall y_0 \in B$ можно соединить путём: 
    существует непрерывное отображением $\gamma: \sqbr{0, 1} = I \rightarrow X$ такое, что $\gamma(0) = x_0$, $\gamma(1) = y_0$, $\gamma(I) \subset X$.
    Тогда получаем, что $\gamma(I) = \br{\gamma(I) \cap A} \cup \br{\gamma(I) \cap B}$, причём $\br{\gamma(I) \cap A}$ и $\br{\gamma(I) \cap B}$ непусты, не пересекаются и открыты в топологии, индуцированной с $X$ на $\gamma(I)$. Значит, $\gamma(I)$ несвязно --- противоречие.
\end{proof}

\begin{nota_bene}
    Обратное неверно. Пример: объединение графика функции $f(x) = \sin \frac{1}{x}, \, x > 0$ с отрезком \\
    $\cbr{\, (0,y) \mid -1 \leq y \leq 1 \,}$. Это подмножество плоскости $\R^2$ связно, но не является линейно связным. Доказательство этого факта остаётся читателю в качестве упражнения.
\end{nota_bene}