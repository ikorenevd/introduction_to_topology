\section{Лекция 1}

Повторение определений из мат. анализа: окрестность точки, открытое множество, замкнутое  множество, непрерывная функция, компакт, связность, метрическое пространство. Так же было отмечено, что топлогию можно задать через систему окрестностей.

\begin{nota_bene}
    $\rho(x,y) = \abs{x - y}$ - метрика, при $x,y \in \R^1$.
\end{nota_bene}

\begin{definition}
    Пара $(X, \rho)$ называется метрическим пространство, если $\rho: X \times X \leftarrow \R_{\geq 0}$ удовлетворяет аксиомам метрики. 
\end{definition}

\begin{theorem}
    $\br{\R^1 , \rho = \abs{x - y}}$ является метрическим пространством.
\end{theorem}

\begin{definition}
    Топологическое пространство $\br{X, \tau}$, где $\tau$ - совокупность подмножеств - тополгия, удовлетвояряющие следуюущим свойствам
    \begin{enumerate}
        \item $\emptyset \in \tau$
        \item $X \in \tau$
        \item $\bigcap_{i = 1}^{k}U_k \in \tau$
        \item $\bigcup U_k \in \tau$
    \end{enumerate}
\end{definition}

\begin{example}
    \begin{enumerate}
        \item антидискретная(тривиальная) топология $\tau = \cbr{\emptyset, X}$
        \item дискретная топология $\tau = 2^X$
        \item $X = \cbr{1, 2}$, способы задания топологии: $\tau_1 = \cbr{X, \emptyset, \cbr{1}}$, $\tau_1 = \cbr{X, \emptyset, \cbr{2}}$
    \end{enumerate}
\end{example}

\begin{definition}
    Пусть $X$ - метрическое пространство. Открытый шар $O_r (x_0) = \cbr{x \in X \ | \ \rho(x, x_0) < r}$
\end{definition}

\begin{definition}
    Пусть $X$ - метрическое пространство. $U \subset X$ - открыто, если $\forall x \in U \ \exists$ окрестность точки(=открытый шар, содержащий $x$) $x$, содержащаяся в $U$. 
\end{definition}

\begin{definition}
    В $X$ произвольном топологическом пространстве $U \subset X$ является замкнутым, если дополнение к нему открыто.
\end{definition}

\begin{example}
    Топология Зарисского определяется в $\C^1$, можно обобщить до $\C^n$. - что-то из алгебраическое геомерии.

    Замкнутое множество - это конечное множество точек.
\end{example}
\begin{exercise}
    Доказать, что это топология.
\end{exercise}

\begin{definition}
    База топологии $\beta \subset \tau \subset 2^X$, если любое открытое множество $U \in \tau$ можно выразить в виде объединения элементов из базы $\beta$, т.е. $U = \bigcup_{\alpha \in A} B_{\alpha}$, где $B_{\alpha}$ - элемент базы.
\end{definition}

База топологии необходима для уменьшения количестав задаваемых открытых множеств для определения топологии.

\begin{lemma}[Достаточное условие на базу топологии]
    Пусть $\beta \subset 2^x$ - набор подмножеств. Если выполняются следующие условия
    \begin{enumerate}
        \item $\forall x \in X \exists B_x \in \beta$ такой, что $x \in B_x$.
        \item $\forall B_1, B_2 \in \beta \Rightarrow (x \in B_1 \cap B_2 \Rightarrow \exists B_3 \in \beta : x \in B_3 \subset B_1 \cap B_2)$ 
    \end{enumerate}
    то $\beta$ является базой топлогии.
\end{lemma}
\begin{proof}
    Вводим всевозможные $U_{\alpha} = \bigcup_{\gamma} B_{\gamma}^{(\alpha)}$. Проверим все свойства из опредления топологии.

    Легко проверить, что выполняются первые 2-а свойства из опредедления топологии. В качество $\emptyset$ можно взять объединение пустого числа множеств, а в качестве $X$ - объединение всех элементов базы, оно будет равно $X$, т.к. для каждого $x \in X$ существует элемент базы, содержащий его.

    Докажем выполнение 3-его свойства. Благодаря принципу математической индукции достаточно доказать, что $k = 2$.
    \[
        U_1 \cap U_2 = \bigcup_{\alpha \in A_1} B_{\alpha}^{(1)} \cap \bigcup_{\alpha \in A_2} B_{\alpha}^{(2)} = \bigcup_{\alpha_1 \in A_1, \alpha_2 \in A_2} B_{\alpha_1}^{(1)} \cap B_{\alpha_2}^{(2)} = \bigcup_{\alpha_1 \in A_1, \alpha_2 \in A_2} \bigcup_{x \in B_{\alpha_1}^{(1)} \cap B_{\alpha_2}^{(2)}} B_{3, x}^{(\alpha_1, \alpha_2)}
    \]

    Тут $B_{3, x}^{(\alpha_1, \alpha_2)}$ существует из-за пункта 2. В итоге мы получили, что $U_1 \cap U_2$ можно выразить в виде объединения элементов базы.
    
    Докажем выполнение 4-го свойства.
    \[
        \bigcup_{\alpha \in A} U_{\alpha} = \bigcup_{\alpha \in A} \bigcap_{i \in I} B_i^{(\alpha)} = \bigcup_{(\alpha, i) \in A \times I} B_i^{\alpha}
    \]
    Опять получили объединения элементов базы.

    Итого всевозможные объединение элементов базы задают топологию на $X$.
\end{proof}

\begin{exercise}
    Повторить доказательство для базы метрического пространства.
\end{exercise}

Можно еще уменьшить количество задаваемых элементов.

\begin{definition}
    $\pi$ называется предбазой топологии, если $\pi \subset \beta \subset \tau \subset 2^X$ и каждое $U$ представляется в виде объединения конечного пересечения элементов предбазы, т.е. $\forall U \in \tau$ выполняется
    \[
        U = \bigcup \bigcap_{i = 1}^k P_i, \text{где } P_i \text{- элемент предбазы}.
    \]
\end{definition}

\begin{nota_bene}
    Любое множество задает предбазу.
\end{nota_bene}

\wip
\begin{example}
    Пусть $X = {1,2,3,4,5}$.

    $\pi = \{\{1,2,3\}, \{2,3,4\}, \{3,4,5\}\}$ - предбаза.

    $\beta = \{\{1,2,3\}, \{2,3,4\}, \{3,4,5\}, \{2,3\}, \{3,4\}, \{3\}, \{1,2,3,4\}, \{2,3,4,5\}, \{\textbf{что-то}\}, \emptyset\}$

    $\tau = \{\ldots, \ldots\}$

    Причем $\pi \subset \beta \subset \tau \subset 2^X$
\end{example}

\begin{definition}
    $f: X \rightarrow Y$ - непрерывная функция, если для каждого открытого $U \subset Y$ выполняется $f^{-1}(U)$ - открыто в $X$.
\end{definition}
