\section{Лекция 1}

Читателю рекомендуется повторить определения окрестности точки, открытого множества, замкнутого множества, непрерывной функции, компакта, связности.

\begin{definition}
    Метрическое пространство --- это пара $(X, \rho)$, где $X$ --- множество, а $\rho: X \times X \rightarrow \R$ --- функция, удовлетворяющая следующим аксиомам:
    \begin{enumerate}
        \item $\forall x,y \in X: \rho(x,y) = 0 \Leftrightarrow x = y$;
        \item $\forall x,y \in X: \rho(x,y) \geq 0$;
        \item $\forall x,y \in X: \rho(x,y) = \rho(y,x)$;
        \item $\forall x,y,z \in X: \rho(x,z) \leq \rho(x,y) + \rho(y,z)$.
    \end{enumerate}
    Функция $\rho$ называется метрикой (функцией расстояния). Часто метрическим пространством называют само множество $X$, если функция $\rho$ очевидно подразумевается.
\end{definition}

\begin{statement}
    $\br{\R^1 , \rho = \abs{x - y}}$ является метрическим пространством.
\end{statement}

\begin{definition}
    Топологическое пространство --- это пара $(X, \mcT)$, где $X$ --- множество, а $\mcT  \subseteq 2^X$ --- набор подмножеств $X$, удовлетворяющий следующим аксиомам:
    \begin{enumerate}
        \item $\es,X \in \mcT$;
        \item $\bigcap_{i = 1}^{n}U_i \in \mcT$, где $U_i \in \mcT \ \ \forall i = 1, \ldots, n$;
        \item $\bigcup_{\alpha \in A} U_{\alpha} \in \mcT$, где $U_{\alpha} \in \mcT \ \ \forall \alpha \in A$ ($A$ --- произвольное индексирующее множество);
    \end{enumerate}
    Множество $\mcT$ называется топологией на $X$, а элементы $\mcT$ --- открытыми подмножествами $X$.
\end{definition}

\begin{example}
    \begin{enumerate}
        \item Антидискретная (тривиальная) топология на любом множестве $X$: \ $\mcT = \cbr{\es, X}$.
        \item Дискретная топология на любом множестве $X$: \ $\mcT = 2^X$.
        \item На $X = \cbr{1, 2}$, можно задать 4 топологии: антидискретную (в таком случае пространство $X$ называется слипшимся двоеточием), дискретную (в таком случае пространство $X$ называется простым двоеточием) и две другие: \ $\mcT_1 = \cbr{X, \es, \cbr{1}}$, $\mcT_2 = \cbr{X, \es, \cbr{2}}$.
        Простраство $X$ с топологиями $\mcT_1$ и $\mcT_2$ называется связным двоеточием.
    \end{enumerate}
\end{example}

\begin{definition}
    Пусть $(X, \rho)$ --- метрическое пространство. Открытый шар в $X$ с центром $x_0$ и радиусом $r$ --- это множество $O_r(x_0) = \cbr{x \in X \ | \ \rho(x, x_0) < r}$. Открытые шары также называют открытыми окрестностями точек, которые они содержат, в метрическом пространстве.
\end{definition}

\begin{definition}
    Пусть $X$ - метрическое пространство. Подмножество $U \subset X$ называется открытым, если $\forall x \in U$ существует открытый шар (= открытая окрестность точки $x$), содержащий $x$ и лежащий в $U$. 
\end{definition}

\begin{nota_bene}
    Любое метрическое пространство является топологическим, если определить топологию на нём через открытые шары (т.е. считать открытые шары открытыми множествами).
\end{nota_bene}

\begin{definition}
    Пусть $X$ - топологическое пространство. Подмножество $U \subseteq X$ называется замкнутым, если $X \backslash U$ открыто.
\end{definition}

\begin{exercise}
    Доказать, что топология может быть определена через понятие замкнутых множеств.
\end{exercise}

\begin{example}
    Топология Зарисского: Рассмотрим множество $\C^1$ и назовём в нём замкнутыми подмножествами любые конечные наборы точек: $\{z_1, \ldots, z_n\}$ (пустой набор точек также считается конечным). \\
    Топологию Зарисского можно обобщить на произвольное множество $X$: будем считать замкнутыми любые конечные подмножества $U \subseteq X$.
    \end{example}
\begin{exercise}
    Доказать, что топология Зарисского действительно является топологией.
\end{exercise}

\begin{definition}
    База $\mfB$ топологии $\mcT$ на X --- это подмножество $\mfB \subseteq \mcT$ такое, что $\forall U \in \mcT$ можно выразить в виде объединения элементов базы $\mfB$, т.е. $U = \bigcup_{\alpha \in A} B_{\alpha}$, где $B_{\alpha} \in \mfB$.
\end{definition}

База топологии позволяет уменьшить количество изначально задаваемых открытых множеств, определяющих топологию.

\begin{lemma}[Достаточное условие на базу топологии]
    Пусть $\mfB \subseteq 2^X$ - набор подмножеств $X$. Тогда если выполняются следующие условия:
    \begin{enumerate}
        \item $\forall x \in X \ \ \exists B_x \in \mfB $: \ $x \in B_x$,
        \item $\forall B_1, B_2 \in \mfB: \ (x \in B_1 \cap B_2 \Rightarrow \exists B_3 \in \mfB : x \in B_3 \subset B_1 \cap B_2)$, 
    \end{enumerate}
    то $\mfB$ является базой некоторой топологии.
\end{lemma}
\begin{proof} %needsbigchange! Нужно переписать доказательство
    Рассмотрим всевозможные $U_{\alpha} = \bigcup_{\gamma} B_{\gamma}^{(\alpha)}$. Проверим все свойства из определения топологии.

    Легко проверить, что выполняются первые 2-а свойства из определения топологии. В качество $\es$ можно взять объединение пустого числа множеств, а в качестве $X$ - объединение всех элементов базы, оно будет равно $X$, т.к. для каждого $x \in X$ существует элемент базы, содержащий его.

    Докажем выполнение 3-его свойства. Благодаря принципу математической индукции достаточно доказать, что $k = 2$.
    \[
        U_1 \cap U_2 = \bigcup_{\alpha \in A_1} B_{\alpha}^{(1)} \cap \bigcup_{\alpha \in A_2} B_{\alpha}^{(2)} = \bigcup_{\alpha_1 \in A_1, \alpha_2 \in A_2} B_{\alpha_1}^{(1)} \cap B_{\alpha_2}^{(2)} = \bigcup_{\alpha_1 \in A_1, \alpha_2 \in A_2} \bigcup_{x \in B_{\alpha_1}^{(1)} \cap B_{\alpha_2}^{(2)}} B_{3, x}^{(\alpha_1, \alpha_2)}
    \]

    Тут $B_{3, x}^{(\alpha_1, \alpha_2)}$ существует из-за пункта 2. В итоге мы получили, что $U_1 \cap U_2$ можно выразить в виде объединения элементов базы.
    
    Докажем выполнение 4-го свойства.
    \[
        \bigcup_{\alpha \in A} U_{\alpha} = \bigcup_{\alpha \in A} \bigcap_{i \in I} B_i^{(\alpha)} = \bigcup_{(\alpha, i) \in A \times I} B_i^{\alpha}
    \]
    Опять получили объединения элементов базы.

    Итого всевозможные объединение элементов базы задают топологию на $X$.
\end{proof}

\begin{exercise}
    Повторить доказательство для базы метрического пространства.
\end{exercise}

\begin{definition} %bigchange! Добавил к определению из лекций согласующееся, проверенное и более простое определение
    Предбаза $\Pi$ топологии $\mcT$ на $X$ --- это множество $\Pi \subset \mfB \subset \mcT$, где $\mfB$ --- база $\mcT$, такое, что: $\forall U \in \mfB$: \ $U$ есть конечное пересечение элементов предбазы, т.е. $\forall U \in \mfB$: \ 
    $U = \bigcap_{i = 1}^k P_i$, где $P_i \in \Pi, k \in \N$. \\
    Иначе говоря: предбаза $\Pi$ топологии $\mcT$ на $X$ --- это множество $\Pi \subset \mfB \subset \mcT$, где $\mfB$ --- база $\mcT$, такое, что: $\forall U \in \mcT$: \ $U$ есть объединение конечных пересечений элементов предбазы, т.е. 
    \[
        \forall U \in \mcT: \ U = \bigcup \bigcap_{i = 1}^k P_i, \text{где } P_i \in \Pi, k \in \N.
    \]
\end{definition}

Предбаза топологии позволяет ещё уменьшить количество изначально задаваемых открытых множеств, определяющих топологию.

\begin{nota_bene}
    Любое множество задает предбазу некоторой топологии.
\end{nota_bene}

\begin{example} %bigchange! Расписал все элементы
    Пусть $X = \{1,2,3,4,5\}$.

    Пусть $\Pi = \{\{1,2,3\}, \{2,3,4\}, \{3,4,5\}\}$ --- предбаза.

    Тогда $\mfB = \{ \underbrace{ \{1,2,3\}, \{2,3,4\}, \{3,4,5\} }_{\textnormal{Элементы } \Pi}, \underbrace{ \{2,3\}, \{3,4\}, \{3\} }_{\textnormal{Все конечные пересечения элементов } \Pi} \}$ --- база,

    $\mcT = \{ \underbrace{ \{1,2,3\}, \{2,3,4\}, \{3,4,5\}, \{2,3\}, \{3,4\}, \{3\} }_{\textnormal{Элементы } \mfB}, \underbrace{ \es, \{1,2,3,4\}, \{2,3,4,5\}, \{1,2,3,4,5\} }_{\textnormal{Все объединения элементов } \mfB} \}$ --- топология на $X$.

\end{example}

%bigchange! Убрал определение непрерывного отображения: на лекции его здесь не было, логичнее дать его во 2 лекции