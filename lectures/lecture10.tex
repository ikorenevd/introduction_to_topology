\section{Лекций 10}

\begin{definition}
    (Вторая аксиома счетности) Говорят, что топологическое пространство $X$ удовлетворяет второй аксиоме счётности (II-AC), если у него есть счётная база.
\end{definition}

\begin{definition}
    Локальная база (база окрестностей) пространства $X$ в точке $x_0 \in X$ --- это набор $\mfB_{x_0}$ окрестностей $O_\alpha(x_0)$ точки $x_0$ такой, что для любой окрестности $U(x_0)$ точки $x_0$ $\exists \alpha: \ O_\alpha(x_0) \subset U$.
\end{definition}

\begin{definition}
    (Первая аксиома счетности) Говорят, что топологическое пространство $X$ удовлетворяет первой аксиоме счётности (I-AC), если $\forall x_0 \in X$ в точке $x_0$ существует счётная локальная база $X$.
\end{definition}

\begin{statement}
    Если $(X, \rho)$ --- метрическое пространство, то $X$ удовлетворяет первой аксиоме счётности. 
\end{statement}
\begin{proof}
    Для любой точки $x_0 \in X$ система $O_r(x_0),\ r \in \mathbb{Q}_{>0}$ является счётной локальной базой в этой точке. 
\end{proof}

\begin{statement}
    Если пространство $X$ удовлетворяет второй аксиоме счётности, то оно сепарабельно.
\end{statement}
\begin{proof}
    Пусть $\mfB$ --- счётная база пространства $X$. Выберем в каждом элементе базы один элемент, получим счётное всюду плотное множество. 
\end{proof}

\begin{statement}
    Из второй аксиомы счётности следует первая аксиома счётности.
\end{statement}
\begin{proof}
    Утверждение очевидно.
\end{proof}

\underline{Вопросы:}
\begin{itemize}
    \item Пример неметризуемого пространства, удовлетворяющей первой аксиоме счетности.
    \item Пример неметризуемого пространства, удовлетворяющей второй аксиоме счетности.
    \item Пример метризуемого пространства без второй аксиомы счетности. (Это $\mathbb{R}$ с дискретной метрикой).
    \item Пример не сепарабельного пространства.
\end{itemize}

\begin{theorem}[Линделёфа]
    Если $X$ удовлетворяет второй аксиоме счетности, то из любого его открытого покрытия можно выбрать не более чем счетное подпокрытие.
\end{theorem}
\begin{proof}
    Введем $\Omega \subset \{U_\alpha\}$, где $\{U_\alpha\}$ -- покрытие. Имеем счетную базу $\{O_1, O_2,...,O_n,...\}$.\\ Тогда $\Omega = \{O_i\ |\ \exists \alpha(i):\ O_i \in U_{\alpha_i}\}$. Так как $\Omega \subset \{O_1, O_2,...,O_n,...\},\ \Omega$ счетное (не более чем счетное).
\end{proof}

\begin{statement}
    $U_{\alpha_i}$ образуют счетное подмножество.
\end{statement}

\begin{theorem}
    Сепарабельное метрическое пространство удовлетворяет второй аксиоме счетности.
\end{theorem}
\begin{proof}
    Счетная база $O_q(s),\ s \in S$ -- счетное, всюду плотное множество, $q \in \mathbb{Q}_{>0}$. 
\end{proof}

\begin{example}
    Прямая Зоргенфрея $=$ стрелка $\mathbb{R}$, база $\{[a, b),\ a,b \in \mathbb{R}\}$. Она сепарабельна, удовлетворяет первой аксиоме счетности, но не удовлетворяет второй аксиоме счетности.
\end{example}

\begin{definition}
    (Непрерывность по Коши) $f: X \mapsto Y$. Пусть $f(x_0) = y_0$, $\forall O(y_0)\ \exists U(x_0):\ f(U(x_0)) \subset O(y_0)$
\end{definition}

\begin{definition}
    (Непрерывность по Гейне) $f: X \mapsto Y$. $\{x_n\} \mapsto x_0 \in X$, при $n \mapsto +\infty$, если $\forall O(x_0)\ \exists N \in \mathbb{R}:\ \forall n \geq N\ x_n \in O(x_n)$.
\end{definition}

\begin{theorem}
    В метрическом пространстве данные определения эквивалентны.
\end{theorem}

\begin{theorem}
    (Урысона о достаточном условии метризуемости) Нормированное пространство, удовлетворяющее второй аксиоме счетности метризуемо.
\end{theorem}

\begin{definition}
    Несвязная сумма -- это топологическое пространство $X \sqcup Y = \{a_x \in X,\ a_y \in Y\}$, $\tau_{X \sqcup Y} = \{u_X, u_Y\}$  
\end{definition}

\begin{definition}
    Декартово произведение топологических пространств -- это $\{(x, y),\ x \in X, y\in Y\}$ с топологией $\tau_{X \times Y} = \{u_X \times u_Y\ |\ u_X \in \tau_X, u_Y \in \tau_Y\}$ -- топология прямого произведения.
\end{definition}

\begin{definition}
    (Топология Тихонова) $\pi_1:X\times Y \mapsto X$ и $\pi_2:X\times Y \mapsto Y$ -- проекции. Тогда предбаза топологии Тихонова: $\pi_1^{-1}(u_x)$ и $\pi_2^{-1}(u_x)$ прообразы открытых в $X$ и $Y$ множеств. 
\end{definition}

\begin{nota_bene}
    Топология Тихонова полезна в случае счетного произведения топологических пространств: $X_1 \times X_2 \times ... \times X_n \times ...$, $\pi_i^{-1}(u_i)$ -- предбаза топологий.
\end{nota_bene}