\section{Лекция 11}

\subsection{Тихоновская топология и теорема Тихонова}

\begin{definition}
    Тихоновская топология - топология на произведение топологических пространств таким образом, что координатные функции непрерывны.
\end{definition}

\begin{nota_bene}
    В конечномерной ситуации Тихоновоская топология совпадает с топологией произведения. В бесконечномерном случае это не так.
\end{nota_bene}

\begin{theorem}[Тихонов]
    $\Pi_{\alpha} X_{\alpha}$, где $X_\alpha$ - компактное пространство. $\Pi_{\alpha} X_{\alpha}$ - компактно в Тихоновской топологии.
\end{theorem}
\begin{proof}
    Возможно будет потом.
\end{proof}

\begin{example}
    Пусть $\cbr{0, 1}$ - несвязное двоеточие. Рассмотрим $\cbr{0, 1}^{\N} = \cbr{0, 1} \times \cbr{0, 1} \times \cbr{0, 1} \times \ldots$.
    
    В топологии Тихонова это гомеоморфно Канторову множеству.
    Можно задать следующим образом
    \[
        f(x_1, x_2, \ldots) = \sum_{k = 1}^{+\infty} \frac{2 x_k}{3^k}
    \]
    \begin{exercise}
        Доказать, что это гомеоморфзим в Тихоновской топологии.
    \end{exercise}


    В топологии произвеедения получится $\tau = 2^{\cbr{0, 1}^{\N}}$.
\end{example}

\begin{example}
    Гильбертов куб(кирпич) = пространство $\sqbr{0, 1}^{\N}$ с тихоновской топологией.
    Он компактен по теореме Тихонова.

    Гильбертов куб гомеоморфен следующему прострнаству
    \[
        \sqbr{0, 1} \times \sqbr{0, \frac{1}{2}} \times \sqbr{0, \frac{1}{3}} \times \ldots
    \]
    А это ряды, причем $x_n \leq \frac{1}{n}$. Причем $\sum_{n = 1 }^{\infty}(x_n)^2 < \infty$.
\end{example}

\subsection{Фактор-топология}

Пусть $\br{X, \tau}$ - топологическое пространство. Пусть на $X$ определено отношение эквивалентности $~$. И тогда $X/~$ - классы эквивалентности.
Если $x$ лежит в $X$, то $\sqbr{x}$ - его класс эквивалентности в $X/~$. Определим топологию $\tau_{~}$ на $X/~$ по следуюущему правилу:
\[
    U \in \tau_{~} \Leftrightarrow \pi^{-1}(U) \in \tau_X
\]

\begin{example}
    Пусть $X = \R^1$, $x ~ y = x - y \in 2\pi Z$.

    Тогда $R^1/~ \rightarrow S^1$

    Аналогично $T^2 = \R^2/~$
\end{example}

\begin{example}
    $X = [0, 1] \subset \R$, если $x ~ y = y - x \in \Q$

    \begin{exercise}
        Доказать, что пространство $X/~$ не хауcдорфовою.
    \end{exercise}
\end{example}

Как понять, что $f: X/~ \rightarrow Y$ непрерывно?
% todo: добавить коммутативную диаграмму
\[
    \widetilde{f}(x) = f(\pi(x))
\]
\begin{theorem}
    $f$ непрерывно тогда и только тогда, когда $\widetilde{f}$ - непрерывна
\end{theorem}
\begin{proof}
    ($\Leftarrow$): Пусть $\widetilde{f}$ - непрерывна, тогда $\widetilde{f}^{-1}(U)$ - открыто в $X$. Из коммутативности диаграммы(формулы выше), следует, что $\pi^{-1}(f^{-1}(U))$ открыто в X. Необходимо доказать, что $V = f^{-1}(U)$ открыто в $Y$, оно открыто в силу определения фактор топологии.

    ($\Rightarrow$): композиция непрерывных.
\end{proof}

\subsection{Склейка пространств}

Рассматриваем $X \bigsqcup Y$. $x_0 \in X$, $y_0 \in Y$.
\begin{definition}
    Букет: $X \bigsqcup Y / x_0 ~ y_0$

    Циллиндр: $X \times I$, где $I$ - отрезок.

    Конус: $Cone(X)$ - стягиваем основание конуса.

    Надстройкка: ???
\end{definition} 

\subsection{Одноточечная компактификация}

\begin{definition}
    Компактификация пространства $X$ = вложение $i$ $X$ в $CX$, где $CX$ - компакт, и топология $X$ индуцирована вложением $i$.
\end{definition}

\begin{definition}[отдноточечтная компактификация Александрова, введена в 1924 году]
    Пусть $CX = X \bigsqcup N$, где $N$ - точка, со следуюищей топологией.

    \[
        \tau_{CX} = \begin{cases}
            U \subset X - \text{открыто} \\
            W = V \bigsqcup N, V \subset X, V = X \setminus K, $K$ - \text{замкнуто компактно} 
        \end{cases}
    \]
\end{definition}

\wip
\begin{theorem}
    $CX$ с определенной топологией явялется компактным пространством.
\end{theorem}
\begin{proof}
    Рассмотрим $K$ - компакт. Пусть $\widetilde{U}_{\alpha} = \cbr{U_{\tilde{\alpha}}, W_{\tilde{\beta}}}$ - покрытие $CX$, в это покрытие точно входит одно $W_{\beta_0} = V_{\beta_0} \bigsqcup N$. $V_{\beta_0} = X \setminus K_{\beta_0}$ - дополнение к компакту.


    % todo: дописать доказательство
\end{proof}