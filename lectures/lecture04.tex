\section{Лекция 4}

\subsection{Компактность.}

\begin{definition}
    Пусть $(X, \mcT)$ --- топологическое пространство. Система $\cbr{U_{\alpha}}_{\alpha \in A} \subseteq \mcT$ открытых подмножеств в $X$ называется (открытым) покрытием множества $Y \subseteq X$, если $Y \subseteq \, \bigcup_{\alpha \in A} U_{\alpha}$.
\end{definition}

\begin{definition}
    Топологическое пространство $X$ называется компактным (иначе, компактом), если из любого его открытого покрытия можно выделить конечное подпокрытие.
\end{definition}

\begin{nota_bene}
    В устаревшей терминологии описанное выше свойство называлось бикомпактностью, а в определении компактности (иначе, счётно-компактности) требовалась счётность исходного покрытия.
\end{nota_bene}

\begin{theorem}
    Отрезок $[a, b]$ компактен.
\end{theorem}
\begin{proof}
    Пусть $\cbr{U_{\alpha}}_{\alpha \in A}$ --- открытое покрытие отрезка $[a, b]$. 
    Рассмотрим множество 
    \[
        \Pi = \{x \in [a, b] \mid [a, x] \ \text{покрывается конечным числом элементов покрытия} \ \{ U_{\alpha} \}\}.
    \]
    Т.о. $\Pi \subseteq [a, b]$, причём $\Pi \neq \es$, т.к. $\exists \, U_{\alpha_0} \in \cbr{U_{\alpha}}_{\alpha \in A} \, : \, a \in U_{\alpha_0}$. Значит, $\Pi$ --- непустое ограниченное подмножество в $\R$, а значит, существует $\sup\Pi$.  
    Обозначим $\veps_0 = \sup\Pi$. Т.к. $\Pi \subseteq [a, b]$, то $\veps_0 \in [a, b]$, значит, $\exists \, U_{\widetilde{\alpha}_0} \in \cbr{U_{\alpha}}_{\alpha \in A} \, : \, \veps_0 \in U_{\widetilde{\alpha}_0}$.
    Но $U_{\widetilde{\alpha}_0}$ открыто, значит, $\exists \, \delta > 0 \, : \, (\veps_0 - \delta, \veps_0 + \delta) \subset U_{\widetilde{\alpha}_0}$.
    Т.к. $\veps_0$ --- супремум множества $\Pi$, то по свойству супремума: $\exists \, x_{\delta} \in \Pi \, : \, x_{\delta} \in \rseg{\veps_0 - \delta, \veps_0}$. Но тогда по определению $\veps_0$:
    $[a, x_{\delta}]$ покрывается конечным набором элементов из $\cbr{U_{\alpha}}_{\alpha \in A}$. Этот набор с добавленным элементом $U_{\widetilde{\alpha}_0}$ является конечным покрытием отрезка $[a, \veps_0]$, значит, $\veps_0 \in \Pi$.

    Предположим теперь, что $\veps_0 < b$. Тогда $\veps_0$ является внутренней точкой отрезка $[a, b]$, а значит, $\exists \, \veps' > 0 \, : \, (\veps_0 - \veps', \veps_0 + \veps') \subset [a, b]$.
    Тогда $[\veps_0 - \frac{\veps'}{2}, \veps_0 + \frac{\veps'}{2}] \subset [a, b]$. По определению $\veps_0: [a, \veps_0 - \frac{\veps'}{2}]$ покрывается конечным набором элементов из $\{ U_{\alpha} \}$.
    Но тогда этот набор с добавленным элементом $(\veps_0 - \veps', \veps_0 + \veps')$ является конечным покрытием отрезка $[a, \veps_0 + \frac{\veps'}{2}]$, а значит, $\veps_0 \neq \sup\Pi$ --- противоречие. 
    Получаем, что $\veps_0 = b$, а значит, весь отрезок $[a, b]$ покрывается конечным числом элементов из $\cbr{U_{\alpha}}_{\alpha \in A}$, т.е. является компактом.
\end{proof}

\begin{lemma}[о вложенных отрезках]
    Пусть дана система вложенных отрезков: $[a_1, b_1] \supset [a_2, b_2] \supset \ldots \supset [a_n, b_n] \supset ...$.
    Тогда их пересечение $\bigcap_{i = 1}^{\infty} [a_i, b_i] \neq \es$. При этом, если $(b_n - a_n) \to 0$ при $n \to \infty$, то их пересечние состоит из одной точки.
\end{lemma}
\begin{proof}
    Данная лемма доказывается в курсе математического анализа, так что здесь её доказательство мы приводить не будем. Однако мы докажем обобщение этой леммы на случай топологических пространств.
\end{proof}

\begin{definition}
    Система $\cbr{X_{\alpha}}_{\alpha \in A}$ подмножеств множества $X$ называется центрированной, если пересечение любого конечного числа её элементов не пусто.
\end{definition}

\begin{lemma}[Обобщение леммы о вложенных отрезках для топологических пространств]
    Пусть $X$ --- топологическое пространство, $\cbr{F_{i}}_{i = 1}^{\infty}$ --- последовательность замкнутых непустых подмножеств $X$ такая, что $X \supset F_1 \supset F_2 \supset \ldots \supset F_{n} \supset \ldots$. 
    Тогда если $X$ --- компакт, то $\bigcap_{i = 1}^{\infty} F_{i} \neq \es$.
\end{lemma}
\begin{proof}
    Данная лемма немедленно следует из следующей теоремы с тем лишь замечанием, что множество замкнуто тогда и только тогда, когда его дополнение открыто.
\end{proof}

\begin{theorem}
    Топологическое пространство $X$ компактно $\Leftrightarrow$ любая центрированная система замкнутых подмножеств в $X$ имеет непустое пересечение. 
\end{theorem}
\begin{proof}
    $(\Rightarrow):$ Пусть $\cbr{F_{i}}_{i = 1}^{\infty}$ --- центрированная система замкнутых подмножеств в $X$ и $\bigcap_{i = 1}^{\infty} F_{i} = \es$. Тогда множества $U_{i} = X \setminus F_{i}$ открыты в $X$.
    Рассмотрим $\bigcup_{i = 1}^{\infty} U_{i}$:

    \[
        \bigcup_{i = 1}^{\infty} U_{i} = \bigcup_{i = 1}^{\infty} (X \setminus F_{i}) = X \setminus \bigcap_{i = 1}^{\infty} F_{i} = X.
    \]
    Значит, система $\cbr{U_{i}}_{i = 1}^{\infty}$ образует покрытие $X$. Т.к. $X$ компактно, то из этого покрытия можно выбрать конечное подпокрытие $\cbr{U_{i_j}}_{j = 1}^{n}$:
    \[
        \bigcup_{j = 1}^{n} U_{i_j} = X.
    \]
    Но тогда $\bigcap_{j = 1}^{n} F_{i_j} = \es$, а значит, система $\cbr{F_{i}}_{i = 1}^{\infty}$ не является центрированной --- противоречие. Значит, $\bigcap_{i = 1}^{\infty} F_{i} \neq \es$.

    $(\Rightarrow):$ Доказательство этого утверждения остаётся читателю в качестве упражнения.
\end{proof}

\begin{definition}
    Топологическое пространство $X$ называется локально компактным, если $\forall x \in X$ и для любой окрестности $O(x)$ точки $x$ существует окрестность $V(x)$ такая, что
    замыкание $\overline{V(x)} \subset O(x)$ и $\overline{V(x)}$ --- компакт.
\end{definition}

\begin{definition}
    Семейство $\cbr{X_{\alpha}}_{\alpha \in A}$ подмножеств в $X$ называется локально конечным, если $\forall x \in X$ существует окрестность $O(x)$ точки $x$, которая пересекается лишь с конечным числом множеств из семейства $\cbr{X_{\alpha}}_{\alpha \in A}$.
\end{definition}

\begin{definition}
    Говорят, что семейство $V$ подмножеств множества $X$ вписано в семейство $U$, если всякий элемент семейства $V$ содержится в некотором элементе семейства $U$.
\end{definition}

\begin{definition}
    Топологичесоке пространство $X$ называется паракомпактным, если в любое его открытое покрытие можно вписать локально конечное подпокрытие.
\end{definition}

\begin{example}
    Пространства $\R^n, n \geq 1$ являются паракомпактными.
\end{example}

\begin{lemma}[о наследовании компактностей]
    Пусть $X$ --- топологическое пространство, $X \supset A$ и $A$ --- замкнуто. Тогда:
    \begin{enumerate}
        \item $X$ компактно $\Rightarrow$ $A$ компактно;
        \item $X$ локально компактно $\Rightarrow$ $A$ локально компактно;
        \item $X$ паракомпактно $\Rightarrow$ $A$ паракомпактно.
    \end{enumerate}
\end{lemma}

\begin{exercise}
    Доказать лемму выше.
\end{exercise}

\begin{statement}
    Пусть $f: X \to Y$ --- непрерывное отображение топологических пространств.
    Тогда если $X$ компактно, то $f(X) \subset Y$ тоже компактно.
\end{statement}
\begin{proof}
    Пусть $\cbr{V_{\alpha}}_{\alpha \in A}$ --- открытое покрытие $f(X)$. Рассмотрим семейство $\cbr{f^{-1}(V_{\alpha})}_{\alpha \in A}$: в силу непрерывности $f$ оно состоит из открытых множеств и является покрытием $X$, а значит, в силу компактности $X$ из него можно выбрать конечное подпокрытие 
    $\cbr{f^{-1}(V_{\alpha_{i}})}_{i = 1}^{n}$. Но тогда семейство $\cbr{V_{\alpha_{i}}}_{i = 1}^{n}$ будет конечным покрытием $f(X)$, а значит, $f(X)$ --- компакт.
\end{proof}

\begin{exercise}
    Рассмотреть аналогичные утверждения о локальной компактности и паракомпактности.
\end{exercise}

\begin{definition}[Аксиомы отделимости]
    Пусть $X$ --- топологическое пространство. Тогда говорят, что $X$ удовлетворяет аксиоме отделимости $T_{i}$ тогда и только тогда, когда:
    \begin{enumerate}
        \item $T_0$ (аксиома Колмогорова): $\forall x, y \in X, x \neq y$: существует окрестность $O(x)$ точки $x$ такая, что $y \notin O(x)$ или существует окрестность $O(y)$ точки $y$ такая, что $x \notin O(y)$.
        \item $T_1$ (аксиома Тихонова): $\forall x, y \in X, x \neq y$: найдутся окрестности $O(x)$ точки $x$ и $O(y)$ точки $y$ такие, что $y \notin O(x)$ и $x \notin O(y)$.
        \item $T_2$ (аксиома Хаусдорфа): $\forall x, y \in X, x \neq y$: найдутся окрестности $O(x)$ точки $x$ и $O(y)$ точки $y$ такие, что $O(x) \cap O(y) = \es$.
        \item $T_3$: для любой точки $x$ из $X$ и для любого замкнутого подмножества $F \subset X$, не содержащего $x$, существуют непересекающиеся окрестности $O(x)$ и $O(F)$ (где окрестность $O(F)$ --- это любое такое подмножество $A \subset X$, что $A \supset F$ и $A \in \mcT$).
        \item $T_4$: для любых $F_1, F_2$ --- замкнутых подмножеств в $X$ таких, что $F_1 \cap F_2 = \es$, существуют непересекающиеся окрестности $O(F_1)$ и $O(F_2)$.
    \end{enumerate}
\end{definition}

\begin{example}
    \begin{enumerate}
        \item Любое пространство с антидискретной топологией не удовлетворяет аксиоме отделимости $T_0$.
        \item Связное двоеточие удовлетворяет аксиоме отделимости $T_0$.
        \item Любое пространство с дискретной топологией удовлетворяет аксиоме отделимости $T_1$.
        \item Любое пространство с антидискретной топологией удовлетворяет аксиоме отделимости $T_3$.
        \item Любое пространство с антидискретной топологией удовлетворяет аксиоме отделимости $T_4$
        \item Рассмотрим пространство $\br{\R, \mcT}$, где $\mcT = \cbr{ \lseg{a, +\infty}_{a \in \R}, \R, \es }$. Оно не удовлетворяет аксиоме отделимости $T_3$, но удовлетворяет аксиоме отделимости $T_4$, т.к.
        все замкнутые множества в нём имеют вид $\R, \es, F_{a} = \br{-\infty, a}, a \in \R$, и для любого замкнутого множества $F_{a}$ существует единственная окрестность $O(F_{a}) = \R$, а значит, $\forall b > a$: отделить $b$ и $F_{a}$ нельзя.
        При этом не существует таких замкнутых множеств $F_{a}$ и $F_{b}$, что $F_{a} \cap F_{b} = \es$.
    \end{enumerate}
\end{example}