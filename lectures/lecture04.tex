\section{Лекция 4}


Компактность

Пусть $X$ - топологическое пространство.
\begin{definition}
    $X$ - компактно, если из любого покрытия можно выделить конечное подпокрытие.

    Раньше это называлось бикомпактностью, а под в компакте требовалось счетность изначального покрытия.
\end{definition}

\begin{example}
    $[a,b]$ - компактен, для доказательстве нужно использовать факт существования точной верхней грани у ограниченности подмножества $\R^1$.
\end{example}
% todo:
\begin{proof}
    % Рассмотрим $[a, b]$. Введем множество 
    % \[
    % P = \cbr{a \in [a, b]: [a, b] \text{покрывается конечным числом множеств из } \cbr{U_{\alpha}})}
    % \]
    % $P$ не пустно, так как $a$ принадлежит $P$.
\end{proof}

\begin{lemma}[О вложенных отрезках]
    Пусть есть система вложенных отрезков $\cbr{[a_n, b_n]}$, где $[a_n, b_n] \subset [a_{n - 1}, b_{n - 1}]$.

    Тогда их пересечение не пусто. Дополнительно, если $|b_n - a_n| \to 0$, тогда их пересечние состоит из одной точки.
\end{lemma}

\begin{definition}
    Центрировання система множеств $\cbr{X_{\alpha} \subset X}$, если пересечение любого конечного числа множеств $X_{\alpha}$ не пустно.
\end{definition}

\begin{lemma}[Обобщение леммы для топологических пространств]
    Пусть $X$ - топологическое пространство, тогда существует последовательность замкнутых не пустых подмножеств $X \supset F_1 \supset F_2 \ldots$ и $\bigcap F_i \neq \emptyset$.
\end{lemma}
\begin{proof}[Доказательство: игра в понятия(определения)]
    Мы знаем, что
    \[
        F_i - \text{замкнуто} \Leftrightarrow U_i = X \setminus F_i \text{- открыто}
    \]
\end{proof}

Лемма вышея является следствием леммы ниже.

\begin{lemma}
    Топологическое пространство компактно $X$ $\Leftrightarrow$ любая центрируемая система замкнутых подмножест имеет непустые пересечение 
\end{lemma}
\begin{proof}
    $(\Rightarrow):$ Пусть $\bigcup_i F_i = \emptyset$, тогда что можно сказать про $\cbr{U_i}$? Рассмотрим $\bigcup_i (X \setminus F_i) = X \setminus \bigcap_i F_i$

    \[
        \bigcup_i U_i \supset X \Rightarrow \text{существует конечное подпокрытие в силу компактности} X
    \]
    \[
        U_{\alpha_1} \cup \ldots U_{\alpha_k} \supset X
    \]
    Следовательно $\cbr{F_i}$ удовлетворяют условию.
\end{proof}

\begin{exercise}
    Доказать утверждение в обратную сторону.
\end{exercise}

\begin{definition}
    $X$ называется локально компактным, если $\forall x \in X$ существует $O(x)$, для которой существует $V(x)$ такая, что 1) Cl(V(x)) $\subset O(x)$; 2) Cl(V(x)) - компактно.
\end{definition}

\begin{definition}
    Семейство подмножеств $X_{\alpha} \subset X$ называется локально конечным, если существует $O(x)$, которая пересекаяется с конечным числом множеств из системы $\cbr{X_{\alpha}}$.
\end{definition}

\begin{definition}
    Топологичесоке пространство $X$ называется паракомпактном, если в любое его открытое покрытие множ вписать локально конечное подпокрытие.
\end{definition}

\begin{example}
    $\R^1$ и $\R^n$ является паракомпактном
\end{example}

\begin{lemma}[наследование компактностей]
    Пусть $X \supset A$, если $A$ - замкнутно, то $A$ сохраняет следующие свойтсва топологического протсрантсва $X$
    \begin{enumerate}
        \item компактно
        \item локально компактно
        \item паракомпактно
    \end{enumerate}
\end{lemma}

\begin{exercise}
    Доказать лемму выше.
\end{exercise}

\begin{statement}
    Пусть $f: X \to Y$ - непрерывное отображение топологических пространств.
    Тогда, если $X$ компактно, тогда $f(X) \subset Y$ тоже компактно.
\end{statement}
\begin{proof}
    Очевидно.
\end{proof}

\begin{exercise}
    Рассмотреть похожие утверждения для локальной компактности и паракомпактности.
\end{exercise}

\begin{definition}[Аксиомы отделимости]
    \begin{enumerate}
        \item $T_0$ (аксиома Колмогорова): $X$ удовлетворяет $T_0$ тогда и только тогда, когда выполнятеся следующиее или существует $O(x)$ такая, что $y \notin O(x)$, или существует $O(y)$ такая, что $x \notin O(y)$ - для каждых двух различных элементов $x, y \in X$.
        \item $T_1$: для двух ралзичных точек найдутся окрестности, удовлетворяющие следующим свойствам $x \notin O(y)$ и $y \notin O(x)$.
        \item $T_2$ (аксиомы Хаусдорфа): для двух различных точек существуют непересекающиеся окрестности.
        \item $T_3$: для любой точки $x$ из $X$ и для любого замкнутого подмножества $F \subset X$, не содержащего $x$, существуют непересекающиеся окрестности $O(x)$ и $O(F)$.
        \item $T_4$: пусть $F_1, F_2$ -- замкнутые множества, причем $F_1 \cap F_2 = \varnothing$. Существуют $O(F_1), O(F_2): O(F_1) \cap O(F_2) = \varnothing$.
    \end{enumerate}
\end{definition}

\begin{exercise}
    Пространство, удовлетворяющие $T_1$, но не удовлетворяющие $T_0$.
\end{exercise}
