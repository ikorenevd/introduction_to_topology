\section{Лекция 7}

\subsection{Разбиение единицы.}

\begin{lemma}[об ужатии]
    Пусть $X$ --- нормальное пространство с конечным покрытием, то есть $X \subset \bigcup_{i = 1}^N U_i$, где $U_i$ --- открытые множества.
    Тогда существует набор открытых множеств $V_i, \ i = 1, \ldots, N$ таких, что: \\
    1) $\forall i: \ \overline{V}_i \subset U_i$, \\ 
    2) $X \subset \bigcup_{i = 1}^{N} V_i$.
\end{lemma}
\begin{proof}
    Доказательство проведём по индукции: будем последовательно рассматривать $U_1, \ldots, U_k$. База индукции при $k = 1$: рассмотрим $U_1$ и множества
    \[
        A = X \setminus (U_2 \cup U_3 \cup \ldots \cup U_N) \text{ и } B = X \setminus U_1.
    \]
    Очевидно, что $A \subset U_1$ и что множества $A$ и $B$ замкнуты. Т.к. $X$ нормально, то существует окрестность $O(A): \overline{O}(A) \cap B = \es$, тогда $\overline{O}(A) \subset U_1$. Положим $V_1 = O(A)$.
    Получаем: $\overline{V_1} \subset U_1$.

    Докажем, что $V_1 \cup U_2 \cup \ldots U_N$ --- покрытие $X$.
    Пусть $x \in X$. Тогда или $x \in A \ \Longrightarrow x \in O(A) = V_1$, или $x \in U_2 \cup U_3 \cup \ldots \cup U_N$.
    Получаем, что $\forall x \in X: x \in V_1 \cup U_2 \cup U_3 \cup \ldots \cup U_N$, значит, данное объединение является покрытием $X$. При этом $\overline{V_1} \subset U_1$. \\
    Индукционный переход: рассмотрим $1 < k \leq N$. Пусть построены  $V_1, \ldots, V_k, U_{k + 1}, \ldots U_N$. Рассмотрим множество
    \[
        A = X \setminus (V_1 \cup \ldots V_k \cup U_{k + 1} \cup \ldots U_N).
    \]
    Далее повторяем рассуждения из случая $k = 1$. Строгое доказательство остаётся читателю в качестве упражнения.
\end{proof}

\begin{definition}
    Пусть $f: X \rightarrow \R$. Носитель функции $f$ --- это множество $\supp{f}$ = $\overline{\cbr{x \in X : f(x) \neq 0}}$, т.е. носитель является замыканием множества тех $x \in X$, на которых функция $f$ не равна нулю.
\end{definition}

\begin{definition}
    Пусть $X$ --- топологическое пространство, $U_1, \ldots, U_N$ --- его конечное покрытие. Тогда набор непрерывных функций $f_i: X \to \R$, $i = 1, \ldots, N$ называется разбиением единицы, подчинённым покрытию $U_1, \ldots, U_N$, если:
    \begin{enumerate}
        \item $\forall i: \supp f_i \subset U_i$,
        \item $\sum_{i = 1}^{N} f_i \equiv 1$ на $X$.
    \end{enumerate}
\end{definition}

\begin{theorem}[о разбиении единицы]
    Пусть $X$ --- нормальное пространство, $U_1, \ldots, U_N$ --- его конечное покрытие. Тогда существует разбиение единицы, подчинённое покрытию $U_1, \ldots, U_N$.
\end{theorem}
\begin{proof}
        По лемме об ужатии: $\forall i = 1, \ldots, N$: в $U_i$ можно вписать $V_i$ такое, что $\overline{V_i} \subset U_i$, при этом набор $V_i$, $i = 1, \ldots, N$ будет являться покрытием $X$.
        По лемме Урысона для $A = \overline{V_i}, \ B = X \setminus U_i$ будет существовать непрерывная функция $\varphi_i: X \to \R$ такая, что $\varphi_i \!\mid_A \ \equiv 1$ и $\varphi_i \!\mid_B \ \equiv 0$.
        Рассмотрим функцию $f_i$, определённую следующим образом:
        \[
            f_i = \frac{\varphi_i}{\sum_{i = 1}^{N}\varphi_i}.
        \]
        Т.к. $\forall x \in X \ \exists \, \overline{V_i}: \ x \in \overline{V_i}$, то $\exists \, \varphi_i: \ \varphi_i(x) = 1$, значит, знаменатель дроби не равен $0$, т.е. функция $f$ корректно определена.
        Очевидно, что $\sum_{i = 1}^{N} f_i \equiv 1$ на $X$. При этом $\overline{V_i} \subset \supp f_i = \supp \varphi_i \subset U_i$. Т.о. получено разбиение единицы на $X$.
\end{proof}

\begin{exercise}
    Доказать, что если $f$ --- непрерывное отображение, то $\supp f$ --- замкнутое множество.
\end{exercise}

\begin{definition}
    Множество $A$ называется всюду плотным в топологическом пространстве $X$, если $\overline{A} = X$.
\end{definition}

\begin{definition}
    Подмножество $A$ в топологическом пространстве $X$ называется нигде не плотным, если $\Int \overline{A} = \es$, т.е. если для каждого непустого открытого $U$ существует открытое $V \subset U$ такое, что $V \cap A = \es$.
\end{definition}

\begin{example}
    \begin{enumerate}
        \item Множество $\Q$ всюду плотно в $\R$.
        \item Множество $\Z$ нигде не плотно в $\R$.
        \item Множество $\R^1$ нигде не плотно в $\R^2$.
    \end{enumerate}
\end{example}

\begin{exercise}
    Доказать, что не существует одновременно и всюду плотное, и нигде не плотное множество в одном пространстве $X$.
\end{exercise}

\begin{definition}
    Топологическое пространство $X$ называется сепарабельным, если в нём содержится счётное всюду плотное подмножество.
\end{definition}

\begin{example}[Канторово множество]
    Канторово множество --- это множество 
    \[
        K = \cbr{x \in [0, 1] \mid x = \sum_{i = 1}^{\infty} \frac{a_i}{3^i}, \ a_i = 0 \text{ или } 2}.
    \]

    Рассмотрим функцию $f: K \to [0,1]$, определённую по следующему правилу:
    \[
        f\br{\sum_{i = 1}^{\infty} \frac{a_i}{3^i}} = \sum_{i = 1}^{\infty} \frac{\widetilde{a}_i}{2^i}, \ \widetilde{a}_i = 
        \begin{cases}
            0, & \text{ если } a_i = 0, \\
            1, & \text{ если } a_i = 2.
        \end{cases}
    \]
\end{example}

\begin{exercise}
    Доказать, что Канторово множество $K$ замкнуто, а приведённая функция $f$ непрерывна.
\end{exercise}

\begin{definition}
    Топологическое пространство $X$ называется совершенным, если в нём нет изолированных точек.
\end{definition}

\begin{exercise}
    Доказать, что Канторово множество совершенно.
\end{exercise}