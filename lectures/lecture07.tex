\section{Лекция 7}

Повторение из прошлой лекции.
\begin{lemma}[Лемма Урысона]
    $X$ - нормальное пространство, $A, B$ - замкнутные непересекающиеся подмножества $X$.
    Тогда существует непрерывная функция $f : X \rightarrow [0,1]$: $f(A) = \cbr{0}, f(B) = \cbr{1}$.
\end{lemma}

\subsection{Разбиение единицы}

\begin{lemma}[об ужатии]
    $X$ - нормальное пространство с конечным покрытием, то есть $X \subset \bigcup_{i = 1}^N U_i$, где $U_i$ - открытое множество.
    Тогда существует набор открытых $V_i, i = 1, \ldots, N$, таких что $\overline{V}_i \subset U_i, i = 1, \ldots, N$ и $X \subset \bigcup V_i$.
\end{lemma}
\begin{proof}[Доказательство (последовательное)]
    Основание $k = 1$, имеем $U_1$. Рассмотрим
    \[
        X \setminus (U_2 \cup \ldots U_N) = A
    \]
    Видно, что $B$ - замкнуто. Очевидно, что $A \subset U_1$.

    Аналогично доказательству лемме Урысона, будет существовать $O(A)$ причем $\overline{O}(A) \subset U_1$, обозначим $V_1 = O(A)$.

    Докажем, что $V_1 \cup U_2 \cup \ldots U_N$ - покрытие $X$.
    Если $x$ лежит в объед $U_i$, $i \geq 2$, то он лежит в $V_1 \cup U_2 \cup \ldots U_N$.
    Если $x$ лежит в $X \setminus (U_2 \cup \ldots U_N) = A \subset V_1$, тогда выполеняется тоже самое.

    Рассмотрим $1 \geq k < N$. Пусть построены  $V_1, \ldots, V_k, U_{k + 1}, \ldots U_N$.
    
    \[
        A' = X \setminus V_1 \cup \ldots V_k \cup U_{k + 1} \cup \ldots U_N
    \]
    
    Слоедовательно мы можем продолжить $k$ до $N$.
\end{proof}

\begin{exercise}
    Подробно расписать доказательство выше.    
\end{exercise}

% todo: оператор \supp
\begin{definition}
    Пусть $f: X \rightarrow \R$. Носитель функции $f$ (обозн $supp f$) = $\cbr{x \in X : f(x) \neq 0}$.
\end{definition}

\begin{definition}
    Пусть $X$ - топологическое пространство, $U_1, \ldots, U_N$ - конечное покрытие, тогда набор непрерывных функций $f_i: X \rightarrow \R$, $i = 1, \ldots, N$ называется разбиением единицы подчиненное покрытию $U_1, \ldots, U_N$, если выполненые два условия:
    \begin{enumerate}
        \item $supp f_i = \overline{V}_i \subset U_i$
        \item $\sum_{i = 1}^{N}f_i = 1$ на $X$
    \end{enumerate}
\end{definition}

\begin{theorem}[о разбиении единицы]
    Пусть $X$ - нормальное пространство, $U_1, \ldots, U_N$ - конечное покрытие, тогда сущесвтует разбиение единицы, подчиненное покрытию $U_1, \ldots, U_N$.
\end{theorem}
\begin{proof}
    \begin{enumerate}
        \item[] По лемме об ужатии, в $U_i$ можно вписать $V_i$ такое, что $\overline{V}_i \subset U_i$. По лемме Урысона для $A = \overline{V}_i, B = X \setminus U_i$, будет существовать непрерывная функция $\varphi_i: [0, 1] \rightarrow \R$ такая, что $\varphi_i(A) = \cbr{1}$ и $\varphi_i(B) = \cbr{0}$, то есть $\varphi_i = 1$ на $A$ и $\varphi_i = 0$ вне $A$.
        Рассмотрим функцию $f_i$ определенную следующим образом
        \[
            f_i = \frac{\varphi_i}{\sum_{i = 1}^{N}\varphi_i}
        \]
        Причем $supp f_i$ зависит от $\varphi_i$, то есть $supp f_i = supp \varphi_i = \overline{V}_i$
    \end{enumerate}
\end{proof}

\begin{exercise}
    Если $f$ - непрерывное отображение, то $supp f$ - замкнутое.
\end{exercise}

\begin{definition}
    $A$ назвается всюду плотным в топологическом пространстве $X$, если $\overline{A} = X$.
\end{definition}

\begin{definition}
    $A$ называется нигде не плотным, если ($int \overline{A} = \emptyset$ - другое определение) для каждого непустого открытого $U$ существует открытое $V \subset U$ такое, что $V \cap U = \emptyset$.
\end{definition}

\begin{definition}
    $X$ - сепарабельно, если в нем существует счетное всюду плотное множество в нем.
\end{definition}

\begin{example}[Канторово множество]
    Если $x = \sum_{i = 1}^{\infty} \frac{a_i}{3^i}$, где $a_i = 0, 2$, то это элемент Канторово множества.

    Можно рассмотреть $f(\sum_{i = 1}^{\infty} \frac{a_i}{3^i}) = \sum_{i = 1}^{\infty} \frac{\widetilde{a}_i}{3^i}$, $\widetilde{a}_i = 1$, если $a_i = 2$, и $\widetilde{a}_i = 0$, если $a_i = 0$.
    % Нужно доказать, что $f$ - непрерывная функция.
\end{example}

\begin{exercise}
    Доказать, что Канторово множество совершенно и доказать непрерывность функции выше.
\end{exercise}

\begin{definition}
    $X$ - совершенное, если не содержить изолированных точек.
\end{definition}

\begin{theorem}[Кривая Пеано]
    Существует непрерывная кривая из отрезка в произведение двух отрезков, т.е. функция $f: I \rightarrow I \times I$, где $I = [0, 1]$
\end{theorem}
